\chapter{Přístrojové vybavení}
\label{sec:instruments}
Na úvod experimentální části je vhodné popsat konkrétní přístroje
a~další součásti používané při pokusech,
neboť některé se opakují ve více aparaturách.

\section{Laser Ekspla \instrname{PL2231-50}}
\label{sec:instruments-laser}
Ústřední součástí všech prováděných pokusů byl samozřejmě laser.
Vždy bylo použito totéž zařízení, konkrétně model \instrname{PL2231-50}
od firmy Ekspla.
Je to diodou čerpaný pevnolátkový pikosekundový laser typu Nd:YAG
poskytující velmi krátké pulzy záření o~vysokém okamžitém výkonu.
Délka pulzu (FWHM) činí \SI{29}{\pico\second}
a~jejich energie může dosahovat až \SI{30}{\milli\joule},
což představuje okamžitý výkon v~řádu \si{\giga\watt}.
Pulzy se opakují s~frekvencí \SI{50}{\hertz}.

\begin{figure}[htp]
	\centering
	\includegraphics[width=\textwidth]{laser}
	\caption{Pikosekundový laser Ekspla (fotografie z~katalogu).
		Použitý model je ze stejné řady.
		Převzato z~\cite{ekspla-datasheet}.}
	\label{fig:instruments-laser}
\end{figure}

Aktivním médiem je krystal yttrito-hlinitého granátu (\ce{Y3Al5O12})
dopovaný ionty neodymu (\ce{Nd3+}).\autocite{wiki-ndyag}
Základní vlnová délka laseru je \SI{1064}{\nano\metre},
stejně jako pro všechny lasery typu Nd:YAG,
jeho součástí jsou ale teplotně stabilizované jednotky obsahující
krystaly di\-hydro\-gen\-fosfo\-rečnanu draselného (KDP a~KD*P),
které umožňují generování druhé a~třetí harmonické frekvence,
tedy vlnových délek \SIlist{532; 355}{\nano\metre}.
\autocite{ekspla-datasheet}

Generování začíná v~hlavním pevnolátkovém oscilátoru čerpaném diodami,
který vytváří řetězce pulzů s~frekvencí opakování jednotlivých pulzů
přibližně \SI{87}{\mega\hertz} (tzv.~\emph{trains}).
Pulzy jsou slabé, jejich energie se pohybuje v~jednotkách \si{\nano\joule}.
Tyto putují do diodového regenerativního zesilovače se zesílením v~řádu $10^6$
a~poté do víceprůchodového výkonového zesilovače,
takže jejich konečná energie je kolem $\SI{30}{\milli\joule}$.
Výstupní energie je nastavitelná v~krocích po zhruba \SI{1}{\percent}
a~je velmi stabilní mezi po sobě jdoucími pulzy
(odchylky jsou menší než \SI{0.5}{\percent} středního kvadratického průměru
při nastavené základní vlnové délce).
\autocite{ekspla-datasheet}

Výstupní svazek je z~\SI{99}{\percent} svisle polarizovaný.
Jeho profil je podle výrobce přibližně gaussovský
(viz obrázek~\ref{fig:instruments-beamprofile})
o~průměru cca \SI{6}{\milli\metre} na úrovni $1/e^2$ maxima
pro vlnovou délku \SI{1064}{\nano\metre}.
\autocite{ekspla-datasheet}
Navzdory těmto specifikacím se ukázalo, že profil svazku je závislý
na nastavené energii pulzu v~míře, kterou nelze zanedbat.
Tato problematika je podrobněji popsána v~dalších kapitolách,
viz především oddíl \ref{sec:lif-rayleigh}.

Laser je vybaven vlastním měřičem, který průběžně zaznamenává energii pulzu.
K~synchronizaci dalších zařízení slouží spouštěcí signál s~nastavitelným
předstihem.\autocite{ekspla-datasheet}

\begin{figure}[htp]
	\centering
	\includegraphics[scale=0.5]{ekspla-beamprofile}
	\caption{Typický profil laserového svazku pro \SI{1064}{\nano\metre}
		v~blízkém poli uvedený v~technickém listu laseru.
		Převzato z~\cite{ekspla-datasheet}.}
	\label{fig:instruments-beamprofile}
\end{figure}

\section{ICCD kamera \instrname{PI-MAX 1024}}
\label{sec:instruments-iccd}
Pro obrazový záznam rychlých dějů s~malou intenzitou záření byla použita
intenzifikovaná CCD kamera (ICCD) od Princeton Instruments.
Šlo o~zakázkově upravený model \instrname{PI-MAX 1024RB-25-FG-43}
se zmenšeným intenzifikátorem pro kratší integrační časy.

Snímačem kamery je CCD čip s~rozlišením \num{1024}\times\SI{256}{\pixel}.
Před snímačem je umístěn intenzifikátor \instrname{Gen II RB},
který je s~čipem spojen prostřednictvím svazku optických vláken.
Intenzifikátor typu \instrname{RB} má široký spektrální rozsah
od přibližně \SI{200}{\nano\metre} do \SI{900}{\nano\metre}
(viz obrázek \ref{fig:instruments-cameraeff}).

Intenzifikátor je zepředu chráněn křemenným vstupním sklem,
za nímž se nachází alkalická fotokatoda.
Jádrem intenzifikátoru je mikrokanálová destička,
která funguje jako fotonásobič každého kanálu
(jeden kanál odpovídá jednomu pixelu na čipu).
Pracovní napětí je kolem \SI{200}{\volt}.
Na vnitřní straně destičky je luminofor P43,
který znásobené elektrony převádí zpět na fotony,
jež putují svazkem optických vláken do CCD čipu.
Čip je vybaven termoelektrickým chlazením pro potlačení temného proudu.

Kamera umožňuje velmi krátké integrační časy pod \SI{2}{\nano\second}
a~záznam obrazu v~dlouhodobě udržitelné frekvenci 50 snímků za sekundu.
Časování je zajištěno vestavěným programovatelným generátorem \instrname{PTG}.

\begin{figure}[htp]
	\centering
	\includegraphics[scale=0.4]{img/pimax-1024}
	\caption{Intenzifikovaná CCD kamera \instrname{PI-MAX 1024}
		(fotografie z~katalogu).
		Převzato z~\cite{pimax-datasheet}.}
	\label{fig:instruments-camera}
\end{figure}

\begin{figure}[htp]
	\centering
	\input{img/cameraeff}
	\caption{Kvantová účinnost kamery deklarovaná výrobcem.
		Podle \cite{pimax-datasheet}.}
	\label{fig:instruments-cameraeff}
\end{figure}

\begin{figure}[htp]
	\centering
	\input{img/camerafilter}
	\caption{Propustnost dvou křemenných skel sloužících jako filtr
		před kamerou.}
	\label{fig:instruments-camerafilter}
\end{figure}
