\chapter*{Úvod}
\addcontentsline{toc}{chapter}{Úvod}
Technologie využívající plazma nacházejí široké uplatnění v~oblastech
vědy a~techniky, medicíně, průmyslu i~jinde.
Pro rozvoj těchto technologií je nezbytné rozumět procesům,
které se v~plazmatu odehrávají, z~čehož plyne potřeba tyto procesy
studovat a~kvantifikovat.

Způsobů zkoumání dějů v~plazmatu je nepřeberné množství,
stejně jako druhů plazmatu.
Důležitou kategorií optických metod jsou laserové metody.
Laser, coby zdroj vysoce koherentního záření velké intenzity,
skýtá mnoho aplikací, které s~jinými světelnými zdroji nejsou proveditelné.
Hojně využívanou skupinou laserů jsou rychlé pulzní lasery,
které produkují velmi krátké pulzy o~vysoké energii.
Jejich hlavní předností je značný výkon:
Energie pulzu je totiž soustředěna do velice krátkého časového intervalu,
což vede k~extrémním hodnotám okamžitého výkonu.
V~případě pikosekundových pulzů je běžné setkávat se s~okamžitým zářivým
výkonem v~řádu gigawattů.
Další výhodou je samotné krátké trvání pulzu, které umožňuje zkoumání
dějů s~vysokým časovým rozlišením.

Motivací pro tuto práci bylo zakoupení nového pikosekundového laseru
pro Ústav fyziky a~technologií plazmatu na Přírodovědecké fakultě
Masarykovy univerzity.
Hlavním záměrem bylo na novém zařízení zprovoznit několik metod
diagnostiky plazmatu, ověřit možnosti laseru a~najít jeho omezení.
Realizace proto obnášela několik nezávislých pokusů,
z~nichž dva jsou v~této práci prezentovány:

Prvním je určení elektrického pole v~dielektrickém bariérovém výboji
v~dusíku za atmosférického tlaku pomocí metody \EFISH{},
která využívá nelineárního optického jevu,
pozorovatelného jen při~vysokém výkonu laseru.
Tento pokus byl na ústavu proveden vůbec poprvé, protože dosavadní
laserové přístroje to neumožňovaly.

Obsáhlejší kapitolou je laserem indukovaná fluorescence selenu
ve vodíkovém difuzním plameni.
Oproti metodě \EFISH{} je tato metoda dobře známá a~zaběhlá.
Její volba vycházela z~praktických požadavků spolupracujícího
Ústavu analytické chemie AVČR, kde má sloužit jako srovnávací
metoda pro určení koncentrace částic atomizovaného analytu.

Posledním provedeným pokusem byla dvoufotonovou absorpcí iniciovaná
fluorescence (TALIF) kryptonu, jejíž realizace však narazila na technické
možnosti aparatury a~dostupného zařízení.
Zmíněný jev sice byl pozorován, ale naměřená data nebyla konzistentní
a~nevedla k~věrohodným výsledkům.
Vzhledem k~časovým možnostem nebyla provedena oprava
a~výsledky zde proto nejsou publikovány.
