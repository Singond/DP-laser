\chapter[\EFISH]{Generování druhé harmonické frekvence
	v~elektrickém poli (\EFISH)}
\label{sec:efishth}

\newcommand\epol{P}
\newcommand\epolsh{P}
\newcommand\esus{\chi}
\newcommand\esusn[1]{\esus^{(#1)}}
\newcommand\eper{\varepsilon}
\newcommand\epervac{\eper_0}
\newcommand\eperrel{\eper_\text{r}}

Mezi novější laserové metody pro diagnostiku plazmatu patří
generování druhé harmonické frekvence v~elektrickém poli (\EFISH{},
z~anglického \emph{electric field induced second harmonic generation}),
vyvinuté na Princeton University a~poprvé popsané
Arthurem Dogariu et al.~v~roce \citeyear{efish-original}.
\autocite{efish-original,efish-2018}
Její podstatou je detekce záření s~dvojnásobnou frekvencí, vznikajícího
v~médiu vystaveném elektrickému poli.
Množství této harmonické složky je funkcí intenzity elektrického pole,
což s~použitím vhodné reference umožňuje tuto intenzitu stanovit.

\section{Princip}
\label{sec:efishth-principle}
Generování druhé harmonické frekvence v~elektrickém poli
je nelineární optický proces třetího řádu.%
\footnote{Pozor na záměnu s~prostým generováním druhé harmonické frekvence,
které je nelineárním procesem \emph{druhého} řádu.}
% TODO: Druhého nebo třetího? V poznámkách mám druhého, Goldberg uvádí třetího.
K~nelineárním optickým jevům dochází, když je porušena lineární závislost
elektrické polarizace $\epol$ na intenzitě elektrického pole,
tedy když přestává platit lineární vztah:
\begin{equation}
	\label{eq:efishth-epol-linear}
	\epol = \epervac (\eperrel - 1) \elfield,
\end{equation}
kde $\epervac$ je permitivita vakua,
$(\eperrel - 1) = \esus$ je elektrická susceptibilita
a~$\elfield$ je intenzita elektrického pole.
Tento vztah je pouhou aproximací platnou pro slabá elektrická pole,
v~případě silného vnějšího pole $\elfield$ nebo vysoké intenzity světla
je proto třeba jej upravit zahrnutím členů vyšších řádů:
\begin{equation}
	\label{eq:efishth-epol-general}
	\vec\epol = \epervac (\esusn1\vec\elfield + \esusn2\vec\elfield^2
		+ \esusn3\vec\elfield^3 + \ldots),
\end{equation}
kde byly zavedeny susceptibility vyšších řádů $\esus{n}$.

\begin{equation}
	\label{eq:efishth-prop}
	\efish \sim A (\alpha^{(3)} \ndens \enlaser \elfield)^2
\end{equation}

\begin{equation}
	\label{eq:efishth-prop-simple}
	\efish \sim (\enlaser \elfield)^2
\end{equation}
