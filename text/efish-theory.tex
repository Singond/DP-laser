\chapter[\EFISH]{Generování druhé harmonické frekvence
	v~elektrickém poli (\EFISH)}
\label{sec:efishth}

\newcommand\epol{P}
\newcommand\epolsh{P^{\mathnormal{(2\angfreq)}}}
\newcommand\esus{\chi}
\newcommand\esusn[1]{\esus^{(#1)}}
\newcommand\eper{\varepsilon}
\newcommand\epervac{\eper_0}
\newcommand\eperrel{\eper_\text{r}}
\newcommand\elfieldext{E_\text{ext}}
\newcommand\elfieldlaser{E^{(\mathnormal\angfreq)}}
\newcommand\efishconst{A}
\newcommand\itylaser{I_\text{laser}}

Mezi novější laserové metody pro diagnostiku plazmatu patří
generování druhé harmonické frekvence v~elektrickém poli (\EFISH{},
z~anglického \emph{electric field induced second harmonic generation}),
vyvinuté na Princeton University a~poprvé popsané
Arthurem Dogariu et al.~v~roce \citeyear{efish-original}.
\autocite{efish-original}
Její podstatou je detekce záření s~dvojnásobnou frekvencí, vznikajícího
v~médiu vystaveném elektrickému poli.
Množství této harmonické složky je funkcí intenzity elektrického pole,
což s~použitím vhodné reference umožňuje tuto intenzitu stanovit.

\section{Princip}
\label{sec:efishth-principle}
Generování druhé harmonické frekvence v~elektrickém poli
je nelineární optický proces třetího řádu.%
\footnote{Pozor na záměnu s~prostým generováním druhé harmonické frekvence,
které je nelineárním procesem \emph{druhého} řádu.}
% TODO: Druhého nebo třetího? V poznámkách mám druhého, Goldberg uvádí třetího.
K~nelineárním optickým jevům dochází, když je porušena lineární závislost
elektrické polarizace $\epol$ na intenzitě elektrického pole,
tedy když přestává platit lineární vztah:
\begin{equation}
	\label{eq:efishth-epol-linear}
	\vec\epol = \epervac (\eperrel - 1) \vec\elfield,
\end{equation}
kde $\epervac$ je permitivita vakua,
$(\eperrel - 1) = \esus$ je elektrická susceptibilita
a~$\elfield$ je intenzita elektrického pole.
Tento vztah je pouhou aproximací platnou pro slabá elektrická pole,
v~případě silného vnějšího pole $\elfield$ nebo vysoké intenzity světla
je proto třeba jej upravit zahrnutím členů vyšších řádů:
\begin{equation}
	\label{eq:efishth-epol-general}
	\vec\epol = \epervac (\esusn1\vec\elfield + \esusn2\vec\elfield^2
		+ \esusn3\vec\elfield^3 + \ldots),
\end{equation}
kde byly zavedeny susceptibility vyšších řádů $\esusn{n}$.

Nelinearita druhého řádu, tj.~přítomnost nenulové $\esusn2$,
vede k~jevům jako je generování součtových a~rozdílových frekvencí,
optické parametrické zesílení (OPA) nebo optické parametrické oscilace (OPO).
Mezi důsledky nelinearity třetího řádu patří například
Kerrův jev, Ramanův rozptyl a~Brillouinův rozptyl.
\autocite{rpphotonics-polarization}

V~médiu se středovou souměrností je susceptibilita druhého řádu $\esusn2$
nulová, proto i~generování druhé harmonické frekvence $2\angfreq$
(které je degenerovaným případem vzniku součtové frekvence), je v~něm nemožné.
Přiložení vnějšího elektrického pole však symetrii naruší
a~umožní druhé harmonické frekvenci vzniknout v~nelineárním procesu třetího
řádu, kde se kromě elektrického pole laseru $\elfieldlaser$ uplatní
i~vnější elektrické pole $\elfieldext$.
\Citeauthor{efish-2018} tento jev popisuje pro případ molekulárního plynu
vztahem:
\begin{equation}
	\label{eq:efishth-epolsh}
	\vec\epolsh = \frac{3}{2} \ndens \esusn3
	(-2\angfreq, 0, \angfreq, \angfreq)
	\vec\elfieldext \vec\elfieldlaser^2,
\end{equation}
kde $\ndens$ je hustota částic plynu,
a~$\esusn3$ je tenzor susceptibility třetího řádu, který závisí
na dipólovém momentu molekul a~orientaci jejich elektrického pole.
\autocite{efish-2018}
Naměřená intenzita druhé harmonické frekvence $\efish$
je úměrná druhé mocnině indukované polarizace $\epolsh$, pročež platí:
\begin{equation}
	\label{eq:efishth-ity}
	\efish = \efishconst \ndens^2 \elfieldext^2 \itylaser^2,
\end{equation}
kde $\efishconst$ je kalibrační konstanta
a~$\itylaser$ je intenzita budicího laseru.
\autocite{efish-2018}
Konstantu $\efishconst$ je možno určit vhodným referenčním měřením
při známém elektrickém poli
a~hustotu částic lze určit jinými způsoby nebo zahrnout
do konstanty $\efishconst$.
Pro určení intenzity elektrického pole je podstatný závěr,
že signál \EFISH{} je přímo úměrný druhé mocnině elektrického pole
i~intenzity budicího laseru:
\begin{equation}
	\label{eq:efishth-prop-simple}
	\efish \propto (\itylaser \elfieldext)^2.
\end{equation}

\section{Provedení}
\label{sec:efishth-setup}
Metoda \EFISH{} byla poprvé provedena s~femtosekundovým laserem,
ale je aplikovatelná i~pro pikosekundové lasery.
Hlavní výhodou delších pulzů pikosekundového laseru je,
že pulzy mají užší spektrální šířku a~lze je tudíž oddělit
od vlastního záření plazmatu vhodným monochromátorem.

\begin{figure}[p]
	\centering
	\includegraphics[width=\textwidth]{efish-setup-general}
	\caption{Schéma jednoduchého experimentálního uspořádání pro měření
		intenzity elektrického pole pomocí \EFISH{}.
		Laser o~základní frekvenci $\angfreq$ je zaostřen do studované
		oblasti, kde dochází vlivem elektrického pole ke vzniku
		druhé harmonické frekvence $2\angfreq$.
		Pomocí dichroického zrcátka a~monochromátoru jsou složky
		odděleny od sebe a~od parazitního signálu
		(například vlastního záření plazmatu),
		takže na výstupu monochromátoru je očištěná druhá harmonická
		složka, jejíž intenzitu měří fotonásobič.}
	\label{fig:efish-setup-general}
\end{figure}

\begin{figure}[p]
	\centering
	\includegraphics{efish-beam}
	\caption{Obvyklá geometrie oblasti vzniku druhé harmonické frekvence.
		Kruhový svazek je zaostřen do zkoumaného místa,
		což umožňuje dobré prostorové rozlišení.
		Signál $2\angfreq$ je koherentní s~budicím svazkem $\angfreq$,
		takže má poměrně vysokou intenzitu,
		ale je integrovaný ve směru svazku.}
	\label{fig:efish-beam}
\end{figure}

Schéma základní instrumentace pro \EFISH{} je
na obrázku~\ref{fig:efish-setup-general}.
Laser produkuje světlo o~frekvenci $\angfreq$.
Volba této frekvence není nijak omezena, takže při ní lze přihlédnout
k~jiným, praktičtějším, důvodům, například možnostem laserového zařízení,
odlišitelnosti signálu od pozadí či citlivosti detektoru.
Metoda se obvykle používá se zaostřeným svazkem,
aby bylo dosaženo vyššího prostorového rozlišení.
Při průchodu médiem vystaveným vnějšímu elektrickému poli dochází
ke vzniku malého množství světla o~druhé harmonické frekvenci $2\angfreq$,
které je koherentní se zdrojovým svazkem a~vystupuje společně s~ním
do detekční soustavy.
Přestože je signál koherentní, a tudíž silnější,
než by byl v~hypotetickém nekoherentním případě,
je o~mnoho řádů slabší než zdrojový svazek, a~je proto nezbytné
tyto dvě složky dobře oddělit.
Největší část zdrojového svazku je oddělena dichroickým zrcadlem
nastaveným tak, aby základní frekvenci propouštěl a~harmonickou odrážel.
