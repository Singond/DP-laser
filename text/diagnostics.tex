\chapter{Laserová diagnostika plazmatu}
\label{sec:diagnostics}

\newcommand\photonflux{\Gamma}

Od svého vynálezu v~roce 1960 je laser používán k~mnoha rozličným účelům,
diagnostiku průhledných prostředí nevyjímaje.
Základní vlastností, která jej odlišuje od ostatních světelných zdrojů,
je tvorba silně koherentního svazku o~vysokém zářivém výkonu.
To jej předurčuje k~mnoha aplikacím, kde je zapotřebí velmi úzký svazek
s~velkým výkonem.

Zvláštní kategorii laserů tvoří rychlé pulzní lasery, tedy zařízení,
která produkují krátké pulzy laserového světla.
Podle délky pulzu se obvykle dělí na nanosekundové, pikosekundové
či femtosekundové.
Rozdělení není jen formální, neboť % TODO

Většinu diagnostických metod lze rozdělit do dvou základních typů.
Nazvěme je metodami absorpčními a~rozptylovými.

\section{Absorpční metody}
\label{sec:diagnostics-absorption}
Společným znakem absorpčních metod je pohlcení fotonů procházejícího
záření částicemi látky, které přitom přecházejí do vyššího energetického
stavu.
Spektrum pohlcených, případně (po předchozí absorpci) vyzářených vlnových
délek je charakteristické pro každou částici, což umožňuje jejich detekci.

\subsection{Prostá absorpce}
\label{sec:diagnostics-absorptionplain}
Nejjednodušší podobou absorpčních metod je měření poklesu spektrální
intenzity světla z~externího zdroje při průchodu zkoumanou látkou.
Pro rozlišení od ostatních metod ji zde nazvěme prostou absorpcí.

Základem metody je externí zdroj záření o~známém spektrálním složení.
Když toto záření prochází látkou, jeho část je pohlcena a~spektrum se změní.
Porovnáním spektra zdroje a~procházejícího záření je možno určit míru
absorpce různých vlnových délek, což vypovídá o~struktuře částic
a~jejich množství.

Označíme-li hustotu fotonového toku $\photonflux$ a~hustotu pohlcujících
částic $\ndens$, lze počet fotonů pohlcených v~elementu objemu vyjádřit
pomocí Beerova-Lambertova zákona jako:
\begin{equation}
	\label{eq:}
	\dd\photonflux = - \photonflux \ndens \xsect \, \dd z,
\end{equation}
kde $\xsect$ je účinný průřez absorpce a~$z$ je směr průchodu záření.
Je-li koncentrace částic $\ndens$ ve~studovaném objemu konstantní,
má rovnice jednoduché řešení ve tvaru exponenciálního poklesu:
\begin{equation}
	\label{eq:}
	\photonflux (z) = \photonflux_0 \eu^{-\ndens\xsect z},
\end{equation}
kde $\photonflux_0$ je hustota fotonového toku v~bodě $z = 0$.
Aplikací tohoto vztahu lze z~poměru původního a~prošlého záření snadno
spočítat absolutní koncentraci pohlcujících částic.

Možnost přímého určení koncentrace částic je hlavní výhodou prosté absorpce.
Její největší nevýhodou je obvykle nízká citlivost, neboť měřený signál
(pokles intenzity) je velmi slabý v~porovnání s~celkovou intenzitou
procházejícího záření.
Dalším omezením je prostorové rozlišení metody, které je možné jen
ve dvou směrech, protože signál je integrován ve směru průchodu svazku.

\subsection{Cavity ring-down spectroscopy}
\label{sec:diagnostics-crds}
Jednou z~možností, jak zvýšit citlivost absorpce, je použít vícenásobný
průchod záření látkou.
Metoda CRDS (angl.~\emph{cavity ring-down spectroscopy})
k~tomu využívá optickou dutinu, tedy soustavu dvou vysoce
odrazných zrcadel naproti sobě.

Při měření je vzorek mezi zracadly ozářen krátkým laserovým pulzem.
Ten se i~po odeznění zdroje mezi zrcadly odráží a~při každém odrazu
ztratí malou část energie, která projde zrcadlem ven.
Měří se rychlost exponenciálního poklesu tohoto signálu v~čase,
z~níž je možno vypočítat absorpční koeficient média v~dutině.

Výhodou metody je velmi vysoká citlivost, protože účinná optická hloub\-ka
se mnohonásobným odrazem prodlouží i~na několik kilometrů.
Druhou výhodou je nezávislost na fluktuaci energie laserových pulzů,
jelikož konstanta exponenciálního poklesu je na celkové energii
pulzu nezávislá.
Nevýhodou metody je především vysoká cena způsobená značnými nároky
kladenými na laserový zdroj i~kvalitu zrcadel.
\autocite{wiki-crds}

\subsection{Laserem indukovaná fluorescence}
\label{sec:diagnostics-lif}
Mezi absorpční metody je možno zařadit také laserem indukovanou fluorescenci
(LIF).
Přestože při fluorescenci je měřeným signálem záření \emph{emitované}
zkoumanými částicemi, absorpce se zde stále uplatňuje při excitaci částic
do zářivého stavu.
O~laserem indukované fluorescenci důkladněji pojednává
kapitola~\ref{sec:lifth}.

% TODO
% \section{Rozptylové metody}
% \label{sec:diagnostics-scattering}
%
% \subsection{Rayleighův rozptyl}
% \label{sec:diagnostics-rayleigh}
% \subsection{Ramanův rozptyl}
% \label{sec:diagnostics-raman}
% \subsection{Thomsonův rozptyl}
% \label{sec:diagnostics-thomson}
% \subsection{Brillouinův rozptyl}
% \label{sec:diagnostics-brillouin}
% \subsection{Comptonův rozptyl}
% \label{sec:diagnostics-compton}

\section{Ostatní metody}
\label{sec:diagnostics-misc}
Některé metody nelze zařadit ani do jedné z~výše uvedených kategorií,
protože spočívají v~jiných fyzikálních principech.
Mezi ně se řadí i~generování druhé harmonické (\EFISH),
jemuž se věnuje kapitola~\ref{sec:efishth}.
