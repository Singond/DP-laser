\chapter*{Závěr}
\addcontentsline{toc}{chapter}{Závěr}
Předmětem práce byly vybrané metody laserové diagnostiky plazmatu
prováděné na nově pořízeném pikosekundovém laseru.
Pozornost byla věnována zejména dvěma metodám,
laserem indukované fluorescenci (LIF)
a~generování druhé harmonické v~elektrickém poli (\EFISH{}).
Uskutečněné pokusy měly zároveň prokázat vhodnost přístroje
k~těmto aplikacím.

Pomocí metody \EFISH{} byl studován dielektrický bariérový Townsendův výboj
v~čistém dusíku za atmosférického tlaku.
Tato poměrně nová metoda umožnila stanovit intenzitu elektrického pole
uvnitř probíhajícího výboje s~výborným časovým i~prostorovým rozlišením.
Dosáhla toho porovnáním intenzity druhé harmonické frekvence vznikající
ve výboji s~intenzitou vzniklou v~referenčním elektrickém poli známé velikosti.
Průběh elektrického pole odpovídal očekávání a~odhalil časový vývoj zkoumaného
výboje.

Druhá část se věnovala aplikaci laserem indukované fluorescence
na potřeby analytické chemie.
Pomocí difuzního vodíkového plamene byl atomizován vzorek obsahující
sloučeninu selenu a~plamen byl vystaven laserovému světlu.
Z~několika určených vlastností vzniklého fluorescenčního záření
byla spočtena absolutní koncentrace částic atomů selenu v~plameni.
Vysoké prostorové rozlišení metody umožnilo popsat,
k~jakým procesům v~plameni dochází.
