\chapter[LIF]{Laserem indukovaná fluorescence v~selenu}

\providecommand\xpos{x}
\providecommand\ypos{y}
\providecommand\laser{L}
\providecommand\lif{F}

\section{Uspořádání experimentu}
\label{sec:lif-setup}

\begin{figure}
	\includegraphics[width=\textwidth]{lif-setup}
	\caption{Uspořádání experimentu.
		Laserový svazek je pomocí válcových rozptylek rozšířen
		do roviného tvaru, který je zboku snímán kamerou.
		Podle \cite{lif-oh}.}
	\label{fig:lif-setup}
\end{figure}

\section{Vyhodnocení}
\label{sec:lif-method}

\begin{figure}
	\input{../lif/results/flame}
	\caption{Snímek plamene bez laseru.
		Kamera byla bez filtru a~nastavena na \num{10000} akumulací.}
	\label{fig:lif-results-flame}
\end{figure}

\begin{figure}
	\input{../lif/results/rayleigh-example}
	\caption{Snímek laserového paprsku ve vzduchu bez plamene.
		Pozorovaný signál je Rayleighův rozptyl.
		Kamera byla bez filtru a~nastavena na \num{100} akumulací.}
	\label{fig:lif-rayleigh-example}
\end{figure}
cc
\subsection{Rayleighův rozptyl}
\label{sec:lif-rayleigh}

\begin{figure}
	\input{../lif/results/rayleigh-profile}
	\caption{Svislý profil laserového svazku pro různé energie pulzu
		určený pomocí Rayleighova rozptylu.}
	\label{fig:lif-rayleigh-profile}
\end{figure}

\begin{figure}
	\input{../lif/results/rayleigh-time}
	\caption{Časový vývoj Rayleighova rozptylu.
		Signál je integrován v~horizontálním směru (ve směry osy $x$).}
	\label{fig:lif-rayleigh-time}
\end{figure}
