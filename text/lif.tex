\chapter[LIF]{Laserem indukovaná fluorescence v~selenu}

\providecommand\xpos{x}
\providecommand\ypos{y}
\providecommand\voigtsigma{\sigma}
\providecommand\voigtgamma{\gamma}

\section{Uspořádání experimentu}
\label{sec:lif-setup}
Měřený vzorek je 10 ppb selenu v roztoku kyseliny chlorovodíkové
o koncentraci \SI{1}{\mol\per\litre}.
Selen proudí spolu s argonem a vodíkem do atomizátoru.

Laserový svazek je na začátku rozdělen děličem svazku.
Odražená část prochází přes dva hranoly pro zvýšení její dráhy nad stolem.
Poté je válcovou rozptylkou rozšířena do svisle rozbíhavého svazku
a~dále pomocí spojky a~druhé válcové rozptylky zaostřena
do svisle orientovaného rovinného svaz\-ku,
který po ořezání clonkou vodorovně prochází zkoumaným plamenem.
Průřez takto upraveného svazku v~oblasti plamene je přibližně obdélník
o~výšce \SI{3}{\milli\metre} a~tloušťce \SI{1}{\milli\metre}.
Schéma celého uspořádání je na obrázku č.~\ref{fig:lif-setup}.

Intenzita svazku je měřena dvojicí detektorů
zaznamenávajích energii jednotlivých pulzů.
Svazek procházející děličem putuje přímo do detektoru
Ophir \instrname{Vega Pyroelectric PE9}
s~rozsahem \SIrange{0.2}{1000}{\micro\joule}.
Za~atomizátorem je umístěn citlivější detektor
Ophir \instrname{Vega Pyroelectric PE9-ES-C}
s~rozsahem \SIrange{0.1}{200}{\micro\joule},
který měří svazek procházející plamenem.
Pro velmi nízké energie pulzu je citlivější detektor přesunut
za dělič (kde je svazek silnější) a~svazek za~atomizátorem není měřen.

Oblast, kde se kříží laserový svazek s~plamenem, je z~boku snímána
ICCD kamerou \instrname{PIMAX-3} od Princeton Instruments.
Před kamerou je dvojice křemenných sklíček sloužící jako filtr.

\begin{figure}[htb]
	\centering
	\includegraphics[width=\textwidth]{lif-setup}
	\caption{Uspořádání experimentu.
		Laserový svazek je pomocí válcových rozptylek rozšířen
		do roviného tvaru, který je zboku snímán kamerou.
		Podle \cite{lif-oh}.}
	\label{fig:lif-setup}
\end{figure}

\section{Vyhodnocení}
\label{sec:lif-method}
Záznamy z~kamery byly korigovány odečtením temného snímku změřeného
s~vypnutým laserem za jinak stejných podmínek a~nastavení.

Pro výpočty byla použita energie laserových pulzů změřená za~atomizátorem.
V~případech, kdy bylo k~dispozici pouze měření za děličem,
byla energie za atomizátorem dopočtena podle závislosti obou energií
v~ostatních měřeních.
% Pro výpočty byla použita energie laserových pulzů změřená za děličem.
Energie byla následně zprůměrována v~časových intervalech odpovídajících
jednotlivým snímkům.
Rozmístění těchto intervalů bylo odhadnuto ze~známé spoušťové
frekvence, počtu akumulací a~vyčítací prodlevy kamery mezi sním\-ky.

Na obrázku č.~\ref{fig:lif-flame} je samotný plamen bez laserové excitace
zaznamenaný ICCD kamerou bez použití filtru.
Intenzita byla v~tomto případě velmi slabá
(plamen byl pouhým okem neviditelný),
snímek je výsledkem \num{10000} akumulací.

Snímek samotného laserového svazku je na obrázku č.~\ref{fig:lif-beam}.
Atomizátor byl vypnut, signál snímku je především Rayleighův rozptyl
svazku na částicích vzduchu.
Snímaná intenzita byla mnohem větší než u~prostého plamene,
uvedený příklad je ze \num{100} akumulací.

Série snímků na obrázku č.~\ref{fig:lif-timeev} ukazuje časový vývoj
fluorescenčního signálu v~průběhu jednoho pulzu.
Atomizátor je v~tomto případě nastaven na obvyklý průtok
a~kamera zaznamenává 100 akumulací.

\begin{figure}[p]
	\centering
	\input{../lif/results/flame}
	\caption{Snímek plamene bez laseru.
		Kamera byla bez filtru a~nastavena na \num{10000} akumulací.}
	\label{fig:lif-flame}
\end{figure}
\begin{figure}[p]
	\centering
	\input{../lif/results/rayleigh-example}
	\caption{Snímek laserového paprsku ve vzduchu bez plamene.
		Pozorovaný signál je Rayleighův rozptyl.
		Kamera byla bez filtru a~nastavena na \num{100} akumulací.}
	\label{fig:lif-beam}
\end{figure}

\begin{figure}[p]
	\centering
	\input{../lif/results/timeev}
	\caption{Typický vývoj signálu LIF v~průběhu pulzu.
		Kamera byla ve všech snímcích nastavena na \num{100} akumulací.
		Hloubka snímané oblasti, daná tloušťkou laserového svazku,
		je zhruba \SI{1}{\milli\metre}.}
	\label{fig:lif-timeev}
\end{figure}

\subsection{Rayleighův rozptyl}
\label{sec:lif-rayleigh}

\begin{figure}
	\centering
	\input{../lif/results/rayleigh-profile}
	\caption{Svislý profil laserového svazku pro různé energie pulzu
		určený pomocí Rayleighova rozptylu.
		Je patrno, že profil je mírně závislý na nastavené energii,
		neboť poloha maximální intenzity se s~rostoucí energií posouvá
		vzhůru (v~grafu doleva).}
	\label{fig:lif-rayleigh-profile}
\end{figure}

\begin{figure}
	\centerline{\input{../lif/results/rayleigh-time}}
	\caption{Časový vývoj Rayleighova rozptylu v~průběhu jednoho pulzu.
		Signál je integrován v~horizontálním směru (ve směry osy $x$).}
	\label{fig:lif-rayleigh-time}
\end{figure}

\subsection{Excitační profil}
\label{sec:lif-excitprof}
Excitační profil byl změřen postupným laděním laseru na vlnové délky
v~rozsahu \SIrange{195.95}{196.20}{\nano\metre}
s~krokem \SI{0.001}{\nano\metre}.
Pokus byl proveden pro několik energií laserového pulzu $\enlaser$,
jak s~prostorovým filtrem, tak bez něj.

Na základě takto změřeného profilu byla poloha maxima LIF předběžně
odhadnuta na \SI{196.032}{\nano\metre}.
Při dalších měřeních byla vlnová délka laseru na\-stavena na tuto hodnotu.

Naměřená data byla dále aproximována Voigtovým profilem pomocí
Le\-ven\-berg-Marquardtova algoritmu.
Detail proložených profilů pro různé energie pulzu s~prostorovým filtrem
je na obrázku č.~\ref{fig:lif-excitprof-fit}
a~optimalizované parametry jsou v~tabulce č.~\ref{tab:lif-excitprof-fit}.

\begin{figure}
	\centering
	\input{../lif/results/excitprof-nofilter}
	\bigskip\par
	\input{../lif/results/excitprof-filter}
	\caption{Excitační profil bez prostorového filtru (nahoře)
		a~s~prostorovým filtrem (dole) pro několik energií pulzu.}
	\label{fig:lif-excitprof-filter}
\end{figure}

\begin{figure}
	\centering
	\input{../lif/results/excitprof-fit}
	\caption{Detail maxima integrálního excitačního profilu
		změřeného s~prostorovým filtrem
		a~spočtených aproximací Voigtovým profilem.}
	\label{fig:lif-excitprof-fit}
\end{figure}

\shorthandoff{-}
\begin{table}[bh]
	\centering
	\caption{Parametry aproximovaných excitačních profilů.
		$\enlaser$ je energie laserového pulzu,
		$\wavelen_\mathrm{max}$ je střed profilu,
		$\voigtsigma$ a~$\voigtgamma$ jsou šířky Gaussova a~Lorentzova profilu
		a~$\sum\Delta\lif^{2}$ je reziduální suma čtverců.}
	\label{tab:lif-excitprof-fit}
	\sisetup{
		table-alignment-mode = format,
		table-number-alignment = center,
	}
	\pgfplotstabletypeset[
		header = false,
		col sep = tab,
		skip first n = 1,
		every head row/.style = {
			before row = \toprule,
			after row = {
				\midrule
			}
		},
		string type,
		columns = {[index] 0, [index] 1, [index] 2, [index] 3, [index] 4},
		skip rows between index = {0}{5},
		columns/0/.style = {
			column name = $\enlaser\ [\si{\micro\joule}]$,
			column type = {S[
				table-format = 1.2,
				round-mode = places,
				round-precision = 2
			]},
		},
		columns/1/.style = {
			column name = $\wavelen_\mathrm{max}\ [\si{\nano\metre}]$,
			column type = {S[
				table-format = 3.4,
				round-mode = places,
				round-precision = 4
			]},
		},
		columns/2/.style = {
			column name = $\voigtsigma\ [\si{\nano\metre}]$,
			column type = {S[
				table-format = 1.4,
				round-mode = places,
				round-precision = 4
			]},
		},
		columns/3/.style = {
			column name = $\voigtgamma\ [\si{\nano\metre}]$,
			column type = {S[
				table-format = 1.4,
				round-mode = places,
				round-precision = 4
			]},
		},
		columns/4/.style = {
			column name = $\sum\Delta\lif^{2}$,
			column type = {S[
				table-format = 3.2,
				round-mode = places,
				round-precision = 2
			]},
		},
		every last row/.style = {after row = \bottomrule}
	]{../lif/results/excitprof-fit.tsv}
\end{table}
\shorthandon{-}

\subsection{Saturace}
\label{sec:lif-saturation}
Fluorescence podle očekávání vykazovala při vyšších intenzitách laseru
znám\-ky saturace.
Jelikož tento jev nebyl zanedbatelný, bylo nutno jej vyšetřit,
aby jeho vliv mohl být zohledněn při stanovení koncentrace.

Laserový svazek měl v~místě plamene rovinný nerozbíhavý tvar
o~tloušťce asi \SI{1}{\milli\metre}.
Teoretická závislost LIF v~rovinném svazku při pozorování zboku
byla odvozena v~\cite{lif-pb}, kde je vyjádřena přibližným vztahem:
\begin{equation}
	\label{eq:lif-saturation}
	\lif(\enlasery) = \frac{2\lifslope}{\lifsat}
	\left( 1 - \frac{\ln(1 + \lifsat\enlasery)}{\lifsat\enlasery} \right).
\end{equation}
Zde $\lif$ je intenzita signálu LIF,
$\enlasery$ je intenzita laserového svazku ve výšce $\ypos$
(předpokládá se vodorovné vedení svazku),
a~$\lifslope$, $\lifsat$ jsou neznámé parametry.
Parametr $\lifslope$ popisuje lineární část závislosti před nasycením,
parametr $\lifsat$ se nazývá saturační parametr a~udává tvar
závislosti v~nasycené oblasti.

Vlnová délka laseru byla držena na konstantní
hodnotě \SI{196.032}{\nano\metre},
zatímco energie pulzu byla postupně nastavována v~rozsahu
od zhruba \SI{0.5}{\micro\joule} do \SI{4.0}{\micro\joule}.

Kvůli rozsahům použitých měřičů bylo nutno měření rozdělit do dvou sérií:
první od \SI{2.0}{\micro\joule} do \SI{4.0}{\micro\joule}
a~druhé od \SI{0.5}{\micro\joule} do \SI{2.0}{\micro\joule}.
Při vyhodnocení se však ukázalo, že data z~obou sérií pro určité místo
ve výboji na sebe nenavazují plynule a~je mezi nimi patrný skok.
Obrázek \ref{fig:lif-saturation-full-example} toto chování ilustruje.
Jako možné vysvětlení se nabízí proměnlivost profilu laserového svazku
při změnách celkové energie pulzu, kterou ani prostorový filtr
nedokázal potlačit úplně.
Tato proměnlivost byla prokázána měřením rozptylu svazku ve~vzduchu,
jak ukazuje obrázek č.~\ref{fig:lif-rayleigh-profile}.

Data v~této podobě tedy nejsou vhodná pro prostorově rozlišené
určení saturace.
Bylo prozkoumáno několik řešení, jak se s~tímto nedostatkem vypořádat.
První možností bylo zohlednit pouze jednu sadu dat.

\begin{figure}
	\centering
	\input{../lif/results/saturation-full-example-index}
	\input{../lif/results/saturation-full-example-1}%
	\input{../lif/results/saturation-full-example-2}
	\input{../lif/results/saturation-full-example-3}%
	\input{../lif/results/saturation-full-example-4}
	\caption{Příklad dat z~měření saturace pro několik vybraných bodů.
		Přehledový snímek odpovídá energii pulzu \SI{3.99}{\micro\joule}.
		Níže jsou průběhy signálu $\lif$ v~závislosti na energii
		laseru $\enlaser$.
		Hodnoty $\lif$ v~nižším a~vyšším intervalu energií na sebe
		dobře nenavazují a~je mezi nimi patrný jistý skok,
		který znesnadňuje proložení předpokládanou
		závislostí \eqref{eq:lif-saturation}.
		Tento nesoulad je pravděpodobně způsoben změnou svislého
		rozložení intenzity laserového svazku při změně energie.}
	\label{fig:lif-saturation-full-example}
\end{figure}
