\chapter[LIF]{Laserem indukovaná fluorescence selenu}

\providecommand\xpos{x}
\providecommand\ypos{y}
\providecommand\voigtsigma{\sigma}
\providecommand\voigtgamma{\gamma}

\section{Uspořádání experimentu}
\label{sec:lif-setup}

\begin{figure}
	\centering
	\begin{tikzpicture}[scale=0.5]
		\seleniumlifgrotrian
	\end{tikzpicture}
	\caption{Excitační schéma selenu použité pro LIF.}
\end{figure}

Měřený vzorek je 10 ppb (tedy \SI{10}{\micro\gram\per\litre})
selenu v roztoku kyseliny chlorovodíkové
o koncentraci \SI{1}{\mol\per\litre}.
Selen proudí spolu s argonem a vodíkem do atomizátoru.

\subsection{Optická část}
Volba optické dráhy se odvíjela od dvou základních požadavků:
Za prvé bylo potřeba eliminovat nebo alespoň dostatečně potlačit
nehomogennost intenzity laserového svazku,
jejíž rozdělení navíc závisí na nastaveném zesílení.
Za druhé byla celková energie laserových pulzů pro účely LIF příliš
velká a~bylo nutné ji před vstupem do plamene značně omezit.

Pro zajištění rovnoměrnějšího rozložení intenzity byl laserový svazek
po výstupu ze~zdroje veden do prostorového filtru tvořeného dvěma
křemennými spojkami a~clonou s~kruhovou dírkou. % TODO: polomer
Následně byl svazek rozdělen děličem svazku.
Jako dělič sloužila skleněná destička,
která většinu záření propouštěla do detektoru
a~pouze malou část odrážela dále do aparatury,
čímž došlo k~požadovanému snížení intenzity.

Kvůli zlepšení prostorového rozlišení metody byl svazek dále ořezán
průchodem svislou štěrbinou na rovinný tvar,
takže konečný průřez svazku byl přibližně obdélník
o~výšce \SI{3}{\milli\metre} a~tloušťce \SI{1}{\milli\metre}.
Takto upravený svazek vodorovně procházel středovou oblastí plamene.
Schéma celého uspořádání je na obrázku č.~\ref{fig:lif-setup}.

Intenzita svazku je měřena dvojicí detektorů
zaznamenávajích energii jednotlivých pulzů.
Svazek procházející děličem putuje přímo do detektoru
Ophir \instrname{Vega Pyroelectric PE9}
s~rozsahem \SIrange{0.2}{1000}{\micro\joule}.
Za~atomizátorem je umístěn citlivější detektor
Ophir \instrname{Vega Pyroelectric PE9-ES-C}
s~rozsahem \SIrange{0.1}{200}{\micro\joule},
který měří svazek procházející plamenem.
Pro velmi nízké energie pulzu je citlivější detektor přesunut
za dělič (kde je svazek silnější) a~svazek za~atomizátorem není měřen.

V~oblasti, kde se křížil laserový svazek s~plamenem, docházelo k~fluorescenci.
Její záření bylo z~boku snímáno ICCD kamerou synchronizovanou s~laserem.
Použitý model kamery byl \instrname{PIMAX-3} od Princeton Instruments.
Citlivost kamery byla zvýšena opakovanou akumulací signálu na čipu.
Počet akumulací pro LIF byl obvykle 100, ale mohl být regulován dle potřeby.
Před kamerou se nacházela dvojice křemenných sklíček,
která sloužila jako filtr pro odstínění parazitního světla.

\begin{figure}[htb]
	\centering
	\includegraphics[width=\textwidth]{img/lif-setup-se}
	\caption{Schéma uspořádání experimentu.
		Svazek je na začátku upraven prostorovým filtrem,
		aby se jeho profil více podobal gaussovskému.
		Pro měření je použita jen malá část svazku odražená na skleněné desce,
		zbytek prochází do~prvního měřiče, jenž zaznamenává jeho energii.
		Svislá štěrbina zajišťuje rovinný profil svazku.}
	\label{fig:lif-setup}
\end{figure}

\section{Vyhodnocení}
\label{sec:lif-method}
Před stanovením koncentrace atomů selenu bylo nutno určit několik
dalších veličin, na nichž výpočet závisí.
Prostřednictvím Rayleighova rozptylu byl zjištěn profil laserového svazku,
který udává prostorové rozložení intenzity excitujícího záření.
Změření excitačního profilu umožnilo nalézt přesnější polohu maxima
fluorescenčního signálu
a~stanovit spektrální překryv laseru a~excitace $\specoverlap$.
Ukázalo se, že fluorescence i~navzdory nízké energii laserových pulzů
projevovala saturaci, kterou bylo tudíž nutno zahrnout do výpočtů.
Podstatným krokem bylo určení doby života zářivého stavu.

Záznam z~ICCD kamery tvoří časová posloupnost jednokanálových sním\-ků,
přičemž každý snímek je tvořen nastaveným počtem akumulací na čipu.
Kamera kromě fluorescence snímá také parazitní signál pocházející
ze zbytkového světla v~laboratoři nebo vlastního záření plamene.
Pro potlačení jejich nežádoucího vlivu
byly všechny snímky korigovány odečtením temného snímku změřeného
s~vypnutým laserem za jinak stejných podmínek a~nastavení.

Pro výpočty byla použita energie laserových pulzů změřená za~atomizátorem.
V~případech, kdy bylo k~dispozici pouze měření za děličem,
byla energie za atomizátorem dopočtena podle závislosti obou energií
v~ostatních měřeních.
% Pro výpočty byla použita energie laserových pulzů změřená za děličem.
Energie byla následně zprůměrována v~časových intervalech odpovídajících
jednotlivým snímkům kamery.
Hranice těchto intervalů byly odhadnuty ze~známé frekvence spouštění
kamery, počtu akumulací na čipu a~vyčítací prodlevy mezi sním\-ky.

Na obrázku č.~\ref{fig:lif-flame} je samotný plamen bez laserové excitace
zaznamenaný ICCD kamerou bez použití filtru.
Intenzita byla v~tomto případě velmi slabá
(plamen byl pouhým okem neviditelný),
snímek je výsledkem \num{10000} akumulací.

Snímek samotného laserového svazku je na obrázku č.~\ref{fig:lif-beam}.
Atomizátor byl vypnut, signál snímku je především Rayleighův rozptyl
svazku na částicích vzduchu.
Snímaná intenzita byla mnohem větší než u~prostého plamene,
uvedený příklad je ze \num{100} akumulací.

Série snímků na obrázku č.~\ref{fig:lif-timeev} ukazuje časový vývoj
fluorescenčního signálu v~průběhu jednoho pulzu.
Atomizátor je v~tomto případě nastaven na obvyklý průtok
a~kamera zaznamenává 100 akumulací.

\begin{figure}[p]
	\centering
	\input{../lif/results/flame}
	\caption{Snímek plamene bez laseru.
		Kamera byla bez filtru a~nastavena na \num{10000} akumulací.}
	\label{fig:lif-flame}
\end{figure}

\subsection{Rayleighův rozptyl}
\label{sec:lif-rayleigh}

\begin{figure}[p]
	\centering
	\input{../lif/results/rayleigh-example}
	\caption{Snímek laserového paprsku ve vzduchu bez plamene.
		Pozorovaný signál je Rayleighův rozptyl.
		Kamera byla bez filtru a~nastavena na \num{100} akumulací.}
	\label{fig:lif-beam}
\end{figure}

\begin{figure}
	\centering
	\input{../lif/results/rayleigh-profile-s}%
	\hfill
	\input{../lif/results/rayleigh-profile-norm-s}
	\caption{Svislý profil laserového svazku pro různé energie pulzu
		určený pomocí Rayleighova rozptylu.
		V~grafu vlevo je změřená intenzita fluorescence sečtená
		ve směru osy $\xpos$.
		Je patrno, že profil je mírně závislý na nastavené energii,
		neboť poloha maximální intenzity se s~rostoucí energií posouvá
		vzhůru (v~grafu doleva).
		Tento posuv je lépe vidět v~pravém grafu, kde jsou profily
		vyhlazené a~normalizované na jedničku.
		Normalizované profily byly použity k~výpočtu svislého
		rozdělení intenzity laserového paprsku pro známé energie pulzu.}
	\label{fig:lif-rayleigh-profile}
\end{figure}

\begin{figure}
	\centerline{\input{../lif/results/rayleigh-time}}
	\caption{Časový vývoj Rayleighova rozptylu v~průběhu jednoho pulzu.
		Signál je integrován v~horizontálním směru (ve směry osy $x$).}
	\label{fig:lif-rayleigh-time}
\end{figure}

\subsection{Excitační profil}
\label{sec:lif-excitprof}
Excitační profil byl změřen postupným laděním laseru na vlnové délky
v~rozsahu \SIrange{195.95}{196.20}{\nano\metre}
s~krokem \SI{0.001}{\nano\metre}.
Pokus byl proveden pro několik energií laserového pulzu $\enlaser$,
jak s~prostorovým filtrem, tak bez něj.

Na základě takto změřeného profilu byla poloha maxima LIF předběžně
odhadnuta na \SI{196.032}{\nano\metre}.
Při dalších měřeních byla vlnová délka laseru na\-stavena na tuto hodnotu.

Naměřená data byla dále aproximována Voigtovým profilem pomocí
Le\-ven\-berg-Marquardtova algoritmu.
Detail proložených profilů pro různé energie pulzu s~prostorovým filtrem
je na obrázku č.~\ref{fig:lif-excitprof-fit}
a~optimalizované parametry jsou v~tabulce č.~\ref{tab:lif-excitprof-fit}.

\begin{figure}
	\centering
	\input{../lif/results/excitprof-nofilter}
	\bigskip\par
	\input{../lif/results/excitprof-filter}
	\caption{Excitační profil bez prostorového filtru (nahoře)
		a~s~prostorovým filtrem (dole) pro několik energií pulzu.}
	\label{fig:lif-excitprof-filter}
\end{figure}

\begin{figure}
	\centering
	\input{../lif/results/excitprof-fit}
	\caption{Detail maxima integrálního excitačního profilu
		změřeného s~prostorovým filtrem
		a~spočtených aproximací Voigtovým profilem.}
	\label{fig:lif-excitprof-fit}
\end{figure}

\shorthandoff{-}
\begin{table}[bh]
	\centering
	\caption{Parametry aproximovaných excitačních profilů.
		$\enlaser$ je energie laserového pulzu,
		$\wavelen_\mathrm{max}$ je střed profilu,
		$\voigtsigma$ a~$\voigtgamma$ jsou šířky Gaussova a~Lorentzova profilu
		a~$\sum\Delta\lif^{2}$ je reziduální suma čtverců.}
	\label{tab:lif-excitprof-fit}
	\sisetup{
		table-alignment-mode = format,
		table-number-alignment = center,
	}
	\pgfplotstabletypeset[
		header = false,
		col sep = tab,
		skip first n = 1,
		every head row/.style = {
			before row = \toprule,
			after row = {
				\midrule
			}
		},
		string type,
		columns = {[index] 0, [index] 1, [index] 2, [index] 3, [index] 4},
		skip rows between index = {0}{5},
		columns/0/.style = {
			column name = $\enlaser\ [\si{\micro\joule}]$,
			column type = {S[
				table-format = 1.2,
				round-mode = places,
				round-precision = 2
			]},
		},
		columns/1/.style = {
			column name = $\wavelen_\mathrm{max}\ [\si{\nano\metre}]$,
			column type = {S[
				table-format = 3.4,
				round-mode = places,
				round-precision = 4
			]},
		},
		columns/2/.style = {
			column name = $\voigtsigma\ [\si{\nano\metre}]$,
			column type = {S[
				table-format = 1.4,
				round-mode = places,
				round-precision = 4
			]},
		},
		columns/3/.style = {
			column name = $\voigtgamma\ [\si{\nano\metre}]$,
			column type = {S[
				table-format = 1.4,
				round-mode = places,
				round-precision = 4
			]},
		},
		columns/4/.style = {
			column name = $\sum\Delta\lif^{2}$,
			column type = {S[
				table-format = 3.2,
				round-mode = places,
				round-precision = 2
			]},
		},
		every last row/.style = {after row = \bottomrule}
	]{../lif/results/excitprof-fit.tsv}
\end{table}
\shorthandon{-}

\subsection{Saturace}
\label{sec:lif-saturation}
Fluorescence podle očekávání vykazovala při vyšších intenzitách laseru
znám\-ky saturace.
Jelikož tento jev nebyl zanedbatelný, bylo nutno jej vyšetřit,
aby jeho vliv mohl být zohledněn při stanovení koncentrace.

Laserový svazek měl v~místě plamene rovinný nerozbíhavý tvar
o~tloušťce asi \SI{1}{\milli\metre}.
Teoretická závislost LIF v~rovinném svazku při pozorování zboku
byla odvozena v~\cite{lif-pb}, kde je vyjádřena přibližným vztahem:
\begin{equation}
	\label{eq:lif-saturation}
	\lif(\enlasery) = \frac{2\lifslope}{\lifsat}
	\left( 1 - \frac{\ln(1 + \lifsat\enlasery)}{\lifsat\enlasery} \right).
\end{equation}
Zde $\lif$ je intenzita signálu LIF,
$\enlasery$ je intenzita laserového svazku ve výšce $\ypos$
(předpokládá se vodorovné vedení svazku),
a~$\lifslope$, $\lifsat$ jsou neznámé parametry.
Parametr $\lifslope$ popisuje lineární část závislosti pro velmi nízké energie,
parametr $\lifsat$ se nazývá saturační parametr a~udává zakřivení
závislosti v~nasycené oblasti.

Vlnová délka laseru byla držena na konstantní
hodnotě \SI{196.032}{\nano\metre},
zatímco energie pulzu byla postupně nastavována v~rozsahu
od zhruba \SI{0.5}{\micro\joule} do \SI{4.0}{\micro\joule}.

Kvůli rozsahům použitých měřičů bylo nutno měření rozdělit do dvou sérií:
první od \SI{2.0}{\micro\joule} do \SI{4.0}{\micro\joule}
a~druhé od \SI{0.5}{\micro\joule} do \SI{2.0}{\micro\joule}.
Při vyhodnocení se však ukázalo, že data z~obou sérií pro určité místo
ve výboji na sebe nenavazují plynule a~je mezi nimi patrný skok.
Obrázek \ref{fig:lif-saturation-full-example} toto chování ilustruje.
Jako možné vysvětlení se nabízí proměnlivost profilu laserového svazku
při změnách celkové energie pulzu, kterou ani prostorový filtr
nedokázal potlačit úplně.
Tato proměnlivost byla prokázána měřením rozptylu svazku ve~vzduchu,
jak ukazuje obrázek č.~\ref{fig:lif-rayleigh-profile}.

\begin{figure}
	\centering
	\input{../lif/results/saturation-full-example-index}
	\input{../lif/results/saturation-full-example-1}%
	\input{../lif/results/saturation-full-example-2}
	\input{../lif/results/saturation-full-example-3}%
	\input{../lif/results/saturation-full-example-4}
	\caption{Příklad dat z~měření saturace pro několik vybraných bodů.
		Přehledový snímek odpovídá energii pulzu \SI{3.99}{\micro\joule}.
		Níže jsou průběhy signálu $\lif$ v~závislosti na energii
		laseru $\enlaser$.
		Hodnoty $\lif$ v~nižším a~vyšším intervalu energií na sebe
		dobře nenavazují a~je mezi nimi patrný jistý skok,
		který znesnadňuje proložení předpokládanou
		závislostí \eqref{eq:lif-saturation}.
		Tento nesoulad je způsoben změnou svislého rozložení intenzity
		laserového svazku při změně celkové energie pulzu.}
	\label{fig:lif-saturation-full-example}
\end{figure}

Určení prostorově rozlišené saturace z~těchto dat se potýkalo s~obtížemi.
Bylo nutno důkladně zohlednit prostorovou variabilitu intenzity svazku.
Z~měření Rayleighova rozptylu svazku ve vzduchu byly k~dispozici
profily svazku naměřené pro čtyři různé celkové energie pulzu,
tyto profily jsou na obrázku \ref{fig:lif-rayleigh-profile} vlevo.

Profily byly pro odstranění šumu vyhlazeny klouzavým průměrem
a~odečtením pozadí.
Pak byly normovány na hodnotu 1 podle vztahu:
\begin{equation}
	\enlaserynorm = \frac{\enlasery}{\int \enlasery \mathrm{d}y},
\end{equation}
což je vidět na obrázku \ref{fig:lif-rayleigh-profile} vpravo.
Normované profily pro mezilehlé energie byly získány lineární interpolací
mezi naměřenými profily (lineární interpolace zachovává normovanost).
Profil intenzity v~každém snímku byl spočten vynásobením příslušného
normovaného profilu celkovou energií pulzu.
Výsledný průběh intenzit je na obrázku
č.~\ref{fig:lif-saturation-full-profile}.

\begin{figure}[htp]
	\centering
	\input{../lif/results/saturation-full-profile}
	\caption{Profil laserového svazku pro různé energie pulzu,
		integrovaný ve směru osy $\xpos$.
		Červené křivky jsou naměřené profily,
		modré křivky byly interpolovány z~naměřených.
		Každá interpolovaná křivka přísluší jednomu snímku kamery.
		Je zřetelný posuv maxima intenzity k~nižším hodnotám $\ypos$
		s~rostoucí energií.}
	\label{fig:lif-saturation-full-profile}
\end{figure}

Závislost intenzity fluorescence $\lif$ na energii laseru $\enlaser$
byla aproximována funkcí \eqref{eq:lif-saturation} pomocí metody
nejmenších čtverců, čímž byly získány parametry $\lifslope$ a~$\lifsat$.
Toto vyhodnocení bylo provedeno zvlášť pro každý pixel snímku,
přičemž každý snímek byl za účelem potlačení šumu vyhlazen klouzavým
průměrem ve čtverci o~rozměru \num{5}\times\SI{5}\pixel.

Spočtené hodnoty fluorescenčního zesílení $\lifslope$ a~saturačního
parametru $\lifsat$ jsou na obrázku~\ref{fig:lif-saturation-full-params}
spolu s~ukázkou aproximovaných průběhů.
Je vidět, že parametr $\lifslope$ je soustředěn do centrální oblasti
nad atomizátorem, zatímco jinde jsou jeho hodnoty velmi nízké.
Za povšimnutí stojí, že směrem vzhůru roste, navzdory předpokladu,
že bude výraznější v~nižší části plamene.

Saturační parametr $\lifsat$ vykazuje při aproximaci větší nestabilitu
a~jeho hodnoty i~rozložení výrazně závisely na provedené korekci profilu
intenzity svazku $\enlasery$.
Na obrázku je jeho hodnota v~oblasti plamene zhruba homogenní:
Je sice patrn nárůst s~rostoucí výškou, stejně jako u~parametru $\lifslope$,
ale ve vodorovném směru se jeví stabilní.

\begin{figure}
	\centering
	\small
	\input{../lif/results/saturation-full-paramsfits}
	\caption{Parametry fluorescence $\lifslope$ a~$\lifsat$
		určené z~korigovaných hodnot energie
		a~příklad aproximovaných závislostí pro několik pixelů.
		Pro potlačení šumu byly snímky z~kamery před zpracováním vyhlazeny
		průměrováním ve čtverci o~rozměrech \num{5}\times\SI{5}{\pixel}.
		Je patrno, že saturační parametr $\lifsat$ je
		v~oblasti plamene ve vodorovném směru přibližně homogenní.}
	\label{fig:lif-saturation-full-params}
\end{figure}

Alternativou k~tomuto postupu je svislé rozlišení vůbec neuvažovat
a~obě veličiny, to jest intenzitu fluorescence $\lif$ i~intenzitu
laseru $\enlasery$, integrovat ve směry $\ypos$.
(V~případě intenzity laserového svazku to znamená vzít přímo
naměřenou hodnoty energie pulzu $\enlaser$.)
Obdobný postup, jako byl popsán výše, pak vede na vodorovně rozlišené
parametry $\lifslope$ a~$\lifsat$ na obrázku~\ref{fig:lif-saturation-x-params}.

\begin{figure}[htp]
	\centering
	\input{../lif/results/saturation-x-params}
	\caption{Parametry $\lifslope$ a~$\lifsat$ spočtené z~dat integrovaných
		ve směru svislé osy $\ypos$.}
	\label{fig:lif-saturation-x-params}
\end{figure}

\subsection{Doba života}
\label{sec:lif-lifetime}

\begin{figure}[p]
	\centering
	\input{../lif/results/timeev}
	\caption{Typický vývoj fluorescenčního signálu v~průběhu jedné periody.
		Časové rozlišení bylo získáno opakovaným měřením mnoha period
		s~postupnou změnou zpoždění kamery za začátkem pulzu.
		Pulz začíná přibližně v~čase $\tim = \SI{5.0}{\nano\second}$.
		Kamera byla ve všech snímcích nastavena na \num{100} akumulací.
		Hloubka snímané oblasti, daná tloušťkou laserového svazku,
		je zhruba \SI{1}{\milli\metre}.}
	\label{fig:lif-timeev}
\end{figure}
