\chapter[LIF]{Laserem indukovaná fluorescence v~selenu}

\providecommand\xpos{x}
\providecommand\ypos{y}
\providecommand\laser{L}
\providecommand\lif{F}

\section{Uspořádání experimentu}
\label{sec:lif-setup}
Měřený vzorek je 10 ppb selenu v roztoku kyseliny chlorovodíkové
o koncentraci \SI{1}{\mol\per\litre}.
Selen proudí spolu s argonem a vodíkem do atomizátoru.

Laserový svazek je na začátku rozdělen děličem svazku.
Odražený svazek prochází přes dva hranoly pro zvýšení jeho dráhy nad stolem.
Poté je válcovou rozptylkou rozšířen do svisle rozbíhavého svazku
a~dále pomocí spojky a~druhé válcové rozptylky zaostřen
do svisle orientovaného rovinného svazku,
který vodorovně prochází zkoumaným plamenem.
Schéma tohoto uspořádání je na obrázku č.~\ref{fig:lif-setup}.

Intenzita svazku je měřena dvojicí detektorů
zaznamenávajích energii jednotlivých pulzů.
Svazek procházející děličem putuje přímo do detektoru
Ophir \instrname{Vega Pyroelectric PE9}
s~rozsahem \SIrange{0.2}{1000}{\micro\joule}.
Za~atomizátorem je umístěn citlivější detektor
Ophir \instrname{Vega Pyroelectric PE9-ES-C}
s~rozsahem \SIrange{0.1}{200}{\micro\joule},
který měří svazek procházející plamenem.
Pro velmi nízké energie pulzu je citlivější detektor přesunut
za dělič (kde je svazek silnější) a~svazek za~atomizátorem není měřen.

Oblast, kde se kříží laserový svazek s~plamenem, je z~boku snímána
ICCD kamerou \instrname{PIMAX-3} od Princeton Instruments.
Před kamerou je dvojice křemenných sklíček sloužící jako filtr.

\begin{figure}[htb]
	\includegraphics[width=\textwidth]{lif-setup}
	\caption{Uspořádání experimentu.
		Laserový svazek je pomocí válcových rozptylek rozšířen
		do roviného tvaru, který je zboku snímán kamerou.
		Podle \cite{lif-oh}.}
	\label{fig:lif-setup}
\end{figure}

\section{Vyhodnocení}
\label{sec:lif-method}
Záznamy z~kamery byly korigovány odečtením temného snímku změřeného
s~vypnutým laserem za jinak stejných podmínek a~nastavení.

Energie laserových pulzů byla zprůměrována v~časových intervalech
příslušejících jednotlivým snímkům.
Rozmístění těchto intervalů bylo odhadnuto ze~známé spoušťové
frekvence, počtu akumulací a~vyčítací prodlevy kamery mezi snímky.

Na obrázku č.~\ref{fig:lif-flame} je samotný plamen bez laserové excitace
zaznamenaný ICCD kamerou bez použití filtru.
Intenzita byla v~tomto případě velmi slabá
(plamen byl pouhým okem neviditelný),
snímek je výsledkem \num{10000} akumulací.

Snímek samotného laserového svazku je na obrázku č.~\ref{fig:lif-beam}.
Atomizátor byl vypnut, signál snímku je především Rayleighův rozptyl
svazku na částicích vzduchu.
Snímaná intenzita byla mnohem větší než u~prostého plamene,
uvedený příklad je ze \num{100} akumulací.

Série snímků na obrázku č.~\ref{fig:lif-timeev} ukazuje časový vývoj
luminiscenčního signálu v~průběhu jednoho pulzu.
Atomizátor je v~tomto případě nastaven na obvyklý průtok
a~kamera zaznamenává 100 akumulací.

\begin{figure}[p]
	\input{../lif/results/flame}
	\caption{Snímek plamene bez laseru.
		Kamera byla bez filtru a~nastavena na \num{10000} akumulací.}
	\label{fig:lif-flame}
\end{figure}
\begin{figure}[p]
	\input{../lif/results/rayleigh-example}
	\caption{Snímek laserového paprsku ve vzduchu bez plamene.
		Pozorovaný signál je Rayleighův rozptyl.
		Kamera byla bez filtru a~nastavena na \num{100} akumulací.}
	\label{fig:lif-beam}
\end{figure}

\begin{figure}[p]
	\input{../lif/results/timeev}
	\caption{Vývoj signálu LIF v~průběhu pulzu.
		Kamera byla ve všech snímcích nastavena na \num{100} akumulací.
		Hloubka snímané oblasti, daná tloušťkou laserového svazku,
		je zhruba \SI{1}{\milli\metre}.}
	\label{fig:lif-timeev}
\end{figure}

\subsection{Rayleighův rozptyl}
\label{sec:lif-rayleigh}

\begin{figure}
	\input{../lif/results/rayleigh-profile}
	\caption{Svislý profil laserového svazku pro různé energie pulzu
		určený pomocí Rayleighova rozptylu.}
	\label{fig:lif-rayleigh-profile}
\end{figure}

\begin{figure}
	\centerline{\input{../lif/results/rayleigh-time}}
	\caption{Časový vývoj Rayleighova rozptylu v~průběhu jednoho pulzu.
		Signál je integrován v~horizontálním směru (ve směry osy $x$).}
	\label{fig:lif-rayleigh-time}
\end{figure}
