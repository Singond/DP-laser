\chapter[\EFISH]{{\EFISH} v~dielektrickém bariérovém výboji}

\newcommand\ypos{y}

\begin{figure}
	\includegraphics[width=\textwidth]{efish-reactor}
	\caption{Uspořádání zkoumaného výboje. Podle \cite{efish-nitrogen}.}
\end{figure}

\begin{figure}
	\includegraphics[width=\textwidth]{efish-setup}
	\caption{Uspořádání celého experimentu.
	První spojka zaostřuje svazek do středu výbojového prostoru
	uvnitř reaktoru.
	Druhá spojka slouží ke kolimaci svazku.
	Dichroická zrcátka odrážejí druhou harmonickou frekvenci
	a~propouštějí základní.
	Energie laserového pulzu je snímána fotodiodou i~měřičem energie.
	Před fotonásobičem je filtr propouštějící pouze úzkou oblast
	kolem druhé harmonické frekvence.
	Podle \cite{efish-nitrogen}.}
\end{figure}

\section{Vyhodnocení}
\label{sec:efish-method}

\begin{figure}[htp]
	\centering
	\input{../efish/results/singleshot}
	\caption{Signál \EFISH{} z~jediného snímku
		a~intenzita laserového pulzu, který jej vyvolal.
		Čas $\tim$ je čas záznamu osciloskopem a~neodpovídá skutečnému času,
		kdy k~ději došlo.
		Zejména není zaručeno, že zpoždění je v~obou případech stejné.}
	\label{fig:efish-singleshot}
\end{figure}

\begin{figure}[htp]
	\centering
	\input{../efish/results/singleshots-compare}
	\caption{Porovnání signálu \EFISH{} pro několik různých snímků.}
	\label{fig:efish-singleshots-compare}
\end{figure}

\begin{figure}
	\sisetup{per-mode = symbol}
	\input{../efish/plots/calib}
	\caption{Určená kalibrační funkce.}
\end{figure}

\begin{figure}[p]
	\sisetup{per-mode = symbol}
	\makebox[\textwidth]{\input{../efish/plots/period-elfield}}
	\caption{Časový vývoj intenzity elektrického pole
		v~různých místech výboje.
		Spodní elektroda se nachází zhruba ve~výšce
		$\ypos = \SI{-0.35}{\milli\metre}$,
		horní elektroda přibližně ve~výšce
		$\ypos = \SI{0.7}{\milli\metre}$.}
\end{figure}
