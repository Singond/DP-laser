\chapter[\EFISH]{{\EFISH} v~dielektrickém bariérovém výboji}

\newcommand\ypos{y}
\providecommand\efishmult{\alpha}
\providecommand\efishshift{P}

\section{Úvod}
\label{sec:efish-intro}
Experiment popsaný v~této kapitole byl jeden z~vůbec prvních pokusů
provedených na novém pikosekundovém laseru po jeho instalaci.
Jeho záměrem bylo pozorovat generování druhé harmonické frekvence
a~využít je k~určení intenzity elektrického pole v~dielektrickém
bariérovém výboji.
%
Druhotným úkolem bylo ověření možností, které laser nabízí.

Výsledky tohto pokusu byly ještě před dokončením práce publikovány
T.~Hoderem et al.~v~článku \cite{efish-nitrogen}, na nějž se dovolujeme
odkazovat.

\section{Uspořádání experimentu}
\label{sec:efish-setup}
Měření bylo uskutečněno na téže aparatuře, jako je popsána
v~\cite{efish-nitrogen},
následující odstavce jsou proto parafrázovány odtamtud.

Předmětem zkoumání byl Townsendův výboj v~dusíku za atmosférického tlaku.
Výboj byl v~dielektrickém bariérovém uspořádání, které ukazuje
obrázek~\ref{fig:efish-reactor}.
Rovinné elektrody o~rozměrech \num{14}\times\SI{15}{\milli\metre}
jsou naneseny na dvou podložních sklíčkách tloušťky \SI{1.1}{\milli\metre}.
Jejich materiálem je sklo s~relativní permitivitou ${\relperm = \num{4.7}}$.
Mezi sklíčky je mezera o~šířce zhruba \SI{1}{\milli\metre}.
Zařízení je umístěno vodorovně v~uzavřeném plastovém reaktoru.
Čistý dusík je přiváděn otvorem v~rovině výboje v~boční stěně reaktoru
a~odváděn podobným otvorem naproti.

Reaktor je vybaven dvěma okénky z~křemenného skla umístěnými naproti
sobě v~úrovni výboje,
která umožňují přímé pozorování a~laserovou diagnostiku.
Okénka jsou skloněna dolů pod Brewsterovým úhlem, aby byl minimalizován
odraz laserového svazku (který je svisle polarizovaný).

Výboj je napájen střídavým napětím o~frekvenci \SI{11}{\kilo\hertz}
a~amplitudě kolem \SI{6}{\kilo\volt},
které je dále modulováno \num{500}\si{\hertz} čtvercovým signálem
se střídou \SI{45}{\percent}.
Výsledkem je deset napěťových cyklů o~celkové délce \SI{909}{\milli\second}
následovaných \SI{1091}{\milli\second} nulového napětí.
\autocite{efish-nitrogen}
Část modulačního cyklu je na obrázku~\ref{fig:efish-overview-full},
detail jednoho napěťového cyklu je na obrázku~\ref{fig:efish-overview-period}.

Schéma optické dráhy je na obrázku~\ref{fig:efish-setup}.
Laser je nastavený na základní frekvenci \SI{1064}{\nano\metre}.
Svazek je zaostřen spojkou o~ohniskové vzdálenosti \SI{50}{\milli\metre}
do středu výbojového prostoru v~reaktoru skrz první okénko.
Mezi elektrodami dochází ke generování druhé harmonické frekvence
(vlnové délky \SI{532}{\nano\metre}), která společně s~původním svazkem
vystupuje druhým okénkem ven.
Zde jsou frekvence od sebe odděleny pomocí dichroického zrcátka,
které harmonickou složku odráží a~základní propouští.
Odražený svazek je po kolimaci druhou spojkou dále očištěn od nežádoucích
složek pomocí druhého dichroického zrcátka a~disperzního hranolu.

Intenzitu harmonického svazku (tedy signál \EFISH{})
měří mikrokanálový fotonásobič
Photek \instrname{PMT210} s~průměrem aktivní oblasti \SI{10}{\milli\metre},
pološířkou pulzu (FWHM) \SI{150}{\pico\second} a~zesílením \num{e6}.
Před fotonásobičem je umístěna irisová clona a~úzkopásmový filtr
pro dodatečné filtrování svazku a~odstínění parazitního záření
z~jiných zdrojů.

Energie původního svazku je měřena dvojím způsobem:
Za prvním dichroickým zrcátkem se nachází pyroelektrický měřič energie
Ophir \instrname{PE50BF-DIF}, jenž snímá celkovou energii pulzů.
Kromě toho je u~vstupního okénka reaktoru umístěna fotodioda
Thorlabs \instrname{DET10A/M}, která měří část svazku odraženou na okénku.

Napájecí napětí výboje, procházející proud, intenzita svazku z~fotodiody
a~signál \EFISH{} z~fotonásobiče jsou zaznamenávány osciloskopem
Keysight \instrname{DSO-S204A},
který je vybaven vysokonapěťovou sondou Tektronix \instrname{P6015A}
a~proudovou sondou Tektronix \instrname{CT2}.

\begin{figure}
	\includegraphics[width=\textwidth]{efish-reactor}
	\caption{Uspořádání zkoumaného výboje.
		Komora reaktoru je válcová, přívod a~odvod plynu se nacházejí
		ve směru kolmém k~ose procházející oběma okénky.
		Šířka mezery mezi podložními sklíčky je \SI{1}{\milli\metre}.
		Podle \cite{efish-nitrogen}.}
	\label{fig:efish-reactor}
\end{figure}

\begin{figure}
	\includegraphics[width=\textwidth]{efish-setup}
	\caption{Uspořádání celého experimentu.
	První spojka zaostřuje svazek do středu výbojového prostoru
	uvnitř reaktoru.
	Druhá spojka slouží ke kolimaci svazku.
	Dichroická zrcátka odrážejí druhou harmonickou frekvenci
	a~propouštějí základní.
	Energie laserového pulzu je snímána fotodiodou i~měřičem energie.
	Před fotonásobičem je filtr propouštějící pouze úzkou oblast
	kolem druhé harmonické frekvence.
	Podle \cite{efish-nitrogen}.}
	\label{fig:efish-setup}
\end{figure}

\begin{figure}[htp]
	\centering
	\input{../efish/results/overview-full}
	\caption{Průběh napětí na elektrodách a~proudu ve výboji
		za několik cyklů.}
	\label{fig:efish-overview-full}
	\vspace{24pt}
	\input{../efish/results/overview-period}
	\caption{Detail jednoho cyklu výboje.}
	\label{fig:efish-overview-period}
\end{figure}

\section{Ověření opakovatelnosti}
\label{sec:efish-check}
Protože laserová aparatura byla v~době měření zcela nová,
bylo vhodné před samotným pokusem ověřit některé charakteristiky
laserových pulzů a~signálu \EFISH{},
zejména opakovatelnost s~konstantní intenzitou a~tvarem,
která je nezbytná k~získání věrohodných výsledků.

Opakovatelnost laserového pulzu byla ověřena změřením sta jednotlivých
snímků (tj.~bez akumulací na osciloskopu) za stálých podmínek.
Laser byl nastaven na konstantní zesílení i~vlnovou délku
a~napájení výboje na konstantní amplitudu napětí.
Intenzita svazku odraženého z~okénka, zachycená fotodiodou,
měla stabilní tvar i~amplitudu, jak ukazuje
obrázek~\ref{fig:efish-pulse-compare}.

Stabilita signálu \EFISH{} byla rovněž dobrá.
Obrázek~\ref{fig:efish-singleshots-compare} ukazuje signál
\EFISH{} změřený fotonásobičem v~průběhu téhož měření,
tedy jednotlivé snímky bez akumulací.
Je vidět, že tvar hlavního pulzu je velmi podobný ve všech snímcích,
šířka je stálá a~náběhová hrana má stejný sklon.
V~některých případech se mírně liší amplituda.

\begin{figure}[htp]
	\centering
	\input{../efish/results/pulse-compare}
	\caption{Ověření konstantnosti laserového pulzu.
		Zobrazena je intenzita laserového svazku odraženého
		na okénku reaktoru zaznamenaná fotodiodou.
		Jedná se o~jednotlivé snímky bez akumulací v~osciloskopu.
		Laser je nastaven na stabilní zesílení.
		Pro přehlednost je zobrazen pouze výběr snímků.}
	\label{fig:efish-pulse-compare}
\end{figure}

\begin{figure}[htp]
	\centering
	\input{../efish/results/singleshots-compare}
	\caption{Ověření reprodukovatelnosti signálu \EFISH{} zachyceného
		fotonásobičem.
		Výběr ze sta snímků zaznamenaných osciloskopem bez průměrování.
		Opakovatelnost se zdá být dostatečná,
		tvar hlavního pulzu je velmi podobný ve všech snímcích.}
	\label{fig:efish-singleshots-compare}
\end{figure}

Vztah mezi signálem z~fotodiody (tj.~intenzitou laseru)
a~fotonásobiče (signálem \EFISH{}) ilustruje
obrázek~\ref{fig:efish-singleshot}.
Je na místě připomenout, že vzájemný časový posun obou signálů
zaznamenaný osciloskopem se~pravděpodobně liší od skutečnosti,
neboť nelze zaručit stejné zpoždění v~obou detekčních cestách.

\begin{figure}[htp]
	\centering
	\input{../efish/results/singleshot}
	\caption{Signál \EFISH{} z~jediného snímku
		a~intenzita laserového pulzu, který jej vyvolal.
		Čas $\tim$ je čas záznamu osciloskopem a~neodpovídá skutečnému času,
		kdy k~ději došlo.
		Zejména není zaručeno, že zpoždění je v~obou případech stejné.}
	\label{fig:efish-singleshot}
\end{figure}

Závislost časového průběhu signálu \EFISH{} na~energii laserového pulzu
je na obrázku~\ref{fig:efish-energy-corrected}.
Je vidět, že tvar je až na absolutní velikost velmi podobný pro všechny
energie.
Malá odchylka se projeví v~poklesové části pulzu, neboť signál způsobený
silnějším pulzem klesá mírně pomaleji než u~slabších.
Tento rozdíl je vidět na obrázku~\ref{fig:efish-energy-norm},
kde jsou data normována tak, aby měla jednotkovou amplitudu signálu.

\begin{figure}[htp]
	\centering
	\input{../efish/results/energy-corrected}
	\caption{Signál \EFISH{} pro různé energie laserového pulzu.}
	\label{fig:efish-energy-corrected}
	\vspace{24pt}
	\input{../efish/results/energy-norm}
	\caption{Signál \EFISH{} pro různé energie pulzu normovaný na stejnou
		amplitudu (\num{-1}).
		Je vidět, že odpovídá tvar hlavního pulzu i~pozadí
		(data jsou korigována odečtením pozadí).
		Největší rozdíly jsou patrny v~poklesové části pulzu.}
	\label{fig:efish-energy-norm}
\end{figure}

\section{Vyhodnocení}
\label{sec:efish-method}
Na začátku zpracování bylo nutno rozhodnout, která charakteristika signálu
\EFISH{} bude použita jako srovnávací.
Nabízely se dvě možnosti: amplituda signálu a~jeho integrál v~čase.
Porovnání obou charakteristik v~datové sadě s~různými energiemi pulzu
(vykreslených na obrázku~\ref{fig:efish-energy-corrected}) prokázalo,
že jsou ekvivalentní, neboť velmi dobře korelují.
Tato korelace je patrna na obrázku~\ref{fig:efish-intmax}.
Při výpočtech byla pro jednoduchost použita amplituda signálu.

\begin{figure}[htp]
	\centering
	\input{../efish/results/intmax}
	\caption{Korelace intenzity signálu \EFISH{} určené dvěma způsoby:
		Jako amplituda intenzity a~jako její integrál.}
	\label{fig:efish-intmax}
\end{figure}

Určení intenzity elektrického pole spočívalo v~porovnání signálu
\EFISH{} z~výboje s~referenčním signálem naměřeným bez výboje,
kdy je možné elektrické pole stanovit jiným způsobem.

Při referenčním měření bylo napětí na elektrodách postupně zvyšováno
z~nuly do zápalného napětí, které se pohybovalo kolem \SI{5}{\kilo\volt}.
Před zapálením výboje bylo elektrické pole mezi elektrodami považováno
za homogenní a~jeho intenzita byla spočtena podle vztahu:
\begin{equation}
	\elfield = \frac{\dbdvoltage}
		{\gapwidth + 2\frac{\barrierthickness}{\relperm}},
\end{equation}
kde $\dbdvoltage$ je napětí na~elektrodách,
$\gapwidth = \SI{1}{\milli\metre}$ je šířka mezery,
$\barrierthickness = \SI{1.1}{\milli\metre}$ je tloušťka každé ze dvou bariér
(podložních sklíček)
a~$\relperm = \num{4.7}$ je jejich relativní permitivita.
Závislost naměřené intenzity na takto spočtené intenzitě elektrického
pole je na obrázku~\ref{fig:efish-period-calib}.

Podle \eqref{eq:efishth-prop-simple} by měl signál \EFISH{} být úměrný
druhé mocnině intenzity elektrického pole $\elfield$.
Naměřená data se však od tohoto vztahu mírně odchylovala a~neprocházela nulou:
I~při nulovém napětí na elektrodách bylo zaznamenáno malé množství
druhé harmonické frekvence.
Jak se později ukázalo, příčinou této odchylky byla okénka reaktoru.
Pevná látka má mnohonásobně vyšší hustotu částic než plyn,
proto i~v~ní docházelo ke slabému generování vyšší frekvence,
přestože zde svazek nebyl zaostřen.

Předpokládaná závislost tedy byla upravena na tvar:
\begin{equation}
	\label{eq:efish-prop}
	\efish(\elfield) \propto \efishmult (\elfield + \efishshift)^2,
\end{equation}
kde $\efishmult$ zahrnuje konstantní násobitele
ze vztahu~\eqref{eq:efishth-prop}
a~$\efishshift$ je nově zavedený posun ve směru $\elfield$.
Tento vztah byl proložen naměřenými daty pomocí metody nejmenších čtverců.
Přepočet signálu \EFISH{} na intenzitu elektrického pole se řídil
proloženou závislostí, která byla protažena i~do záporných hodnot $\elfield$.

Přepočetní funkce byla určena několikrát v~průběhu hlavního měření.
Poprvé v~počáteční poloze svazku mezi elektrodami ($\ypos = 0$),
podruhé v~dolní poloze svazku ($\ypos = \SI{-0.35}{\milli\metre}$)
a~potřetí opět v~počáteční poloze ($\ypos = 0$) před měřením elektrického
pole ve vyšších polohách.
Všechny tři spočtené přepočetní funkce jsou
na obrázku~\ref{fig:efish-period-calib}.
Při přepočtu byla používána levá nebo pravá větev každé funkce
podle předpokládaného směru neznámého elektrického pole.

\section{Výsledky}
\label{sec:efish-results}

\begin{figure}
	\centering
	\input{../efish/results/period-calib-bilateral}
	\caption{Kalibrační funkce použité pro zpětný přepočet signálu \EFISH{}
		($\efish$) na intenzitu elektrického pole.
		Funkce ve středu mezery byla změřena dvakrát, poprvé před měřením
		elektrického pole v~dolní části mezery
		(označeného jako \emph{střed~1})
		a~podruhé před měřením horní části (\emph{střed 2}).
		Záporná větev je extrapolována z~pravé.
		Označená minima ukazují, že všechny funkce jsou mírně posunuty
		doleva (zhruba o~\SI{0.4}{\mega\volt\per\metre}).}
	\label{fig:efish-period-calib}
\end{figure}

\begin{figure}[htp]
	\centering
	\input{../efish/results/period-efish}
	\caption{Intenzita \EFISH{} zaznamenaná v~různých výškách
		v~průběhu jedné periody budicího napětí.}
	\label{fig:efish-period-efish}
\end{figure}

\begin{figure}[p]
	\sisetup{per-mode = symbol}
	\input{../efish/results/period-elfield}
	\caption{Časový vývoj intenzity elektrického pole
		v~různých místech výboje.
		Spodní elektroda se nachází zhruba ve~výšce
		$\ypos = \SI{-0.35}{\milli\metre}$,
		horní elektroda přibližně ve~výšce
		$\ypos = \SI{0.7}{\milli\metre}$.}
\end{figure}

\begin{figure}[htp]
	\centering
	\input{../efish/results/period-amplitude}
	\caption{Amplituda intenzity elektrického pole určená z~maxima
		a~minima spočteného průběhu v~čase.
		Měření proběhlo ve dvou sériích (kladné a~záporné $\ypos$),
		které se překrývají v~bodě $\ypos = 0$.
		To je příčinou dvojí hodnoty v~tomto bodě.}
	\label{fig:efish-period-amplitude}
\end{figure}
