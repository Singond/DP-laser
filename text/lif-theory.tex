\chapter[LIF]{Laserem indukovaná fluorescence}
\label{sec:lifth}
Počátky laserem indukované fluorescence (LIF) sahají do roku 1968,
kdy Richard Zare pomocí \num{632.8}\si{\nano\metre} čáry helium-neonového
laseru analyzoval částice v~draslíkových výparech.
\autocite{lif-original}
Od té doby se s~úspěchem používá ke studiu průhledných médií,
jako jsou plameny nebo plazma.
Ve fyzice plazmatu hraje klíčovou roli při detekci reaktivních částic.
% TODO: \autocite{dvorak1}

Metoda v~sobě spojuje prvky absorpční a~emisní spektroskopie.
Je sice obecně složitejší a~vyžaduje rozsáhlejší a~dražší přístrojové vybavení
než obě tyto metody, ale na oplátku poskytuje několik zásadních výhod.
Mezi nejdůležitější patří nízký detekční limit
a~schopnost detekce nezářivých částic nebo částic s~krátkou dobou života.
Velmi užitečné je také vysoké prostorové rozlišení metody:
Pokud je laserový svazek zaostřen do úzkého profilu a~pozorován z~boku,
je možné získat signál rozlišený ve všech třech prostorových rozměrech.
Při použití krátkopulzního laseru navíc umožňuje výborné časové rozlišení
v~řádu nanosekund až femtosekund.
\autocite{lif-pb}

\section{Princip}
Základní myšlenka metody je pozorování fluorescenčního záření vznikajícího
při deexcitaci zkoumaných částic ze stavu vybuzeného absorpcí laserového
světla.
Obecné schéma nejjednodušší varianty, čítající pouhé tři hladiny,
je na obrázku \ref{fig:lifth-levels}.
Atomy či jiné částice jsou dopadem laseru excitovány ze~základní hladiny~1
do excitovaného stavu~3.
Tento stav je depopulován samovolnou emisí zpět do nižších stavů
(v~tříhladinovém modelu do hladin 1 a~2),
vynucenou emisí do základního stavu způsobenou laserovým zářením
a~řadou dalších procesů probíhajících při srážkách s~okolními částicemi,
které je možno souhrnně nazvat zhášením.
Záření přechodu z~hladiny~3 na hladinu~2 je snímaný fluorescenční signál.

\begin{figure}
	\centering
	\begin{tikzpicture}[scale=0.5]
		\small
		\lifgrotrian
	\end{tikzpicture}
	\caption{Obecné excitační schéma jednofotonové LIF.
		Parametry $\einsteina{i}{j}$ a~$\einsteinb{i}{j}$ jsou Einsteinovy
		koeficienty a~$I$ je intenzita záření.
		Kromě samovolné emise jsou vyšší stavy depopulovány zhášením
		(koeficienty $Q_{ij}$).
		Podle \cite{lif-pb}.}
	\label{fig:lifth-levels}
\end{figure}

\begin{figure}[htb]
	\centering
	\includegraphics[width=\textwidth]{lif-setup-general}
	\caption{Příklad uspořádání experimentu s~LIF.
		Laserový svazek je v~tomto případě pomocí válcových rozptylek
		rozšířen do rovinného tvaru, který je zboku snímán kamerou.
		Podle \cite{lif-oh}.}
	\label{fig:lifth-setup}
\end{figure}

\section{Určení koncentrace částic}
\label{sec:lifth-concentration}
\providecommand\vol{V}
\providecommand\sensabs{D_\text{a}}
\providecommand\lifsens{D_\text{F}}
\providecommand\rayleighsens{D_\text{R}}
\providecommand\lifsignal{M_\text{F}}
\providecommand\rayleighsignal{M_\text{R}}
\providecommand\lifeff{\qeff_\text{F}}
\providecommand\rayleigheff{\qeff_\text{R}}
\providecommand\rayleighdxsect{\dv{\sigma_\text{R}}{\solidangle}}
\providecommand\rayleighndens{\ndens_\text{R}}
\providecommand\enlaserrayleigh{L_\text{R}}
\providecommand\beamprofile{s}
\providecommand\quenching{Q}
\providecommand\liftotal{F}
Velkou výhodou LIF je možnost určit absolutní koncentraci detekovaných částic,
neboť intenzita fluorescence je funkcí této koncentrace.
Závislost je v~limitě nízkých energií laseru lineární,
s~rostoucí energií se pak objevují účinky saturace,
které fluorescenci oslabují (viz dále v~oddíle \ref{sec:lifth-saturation}).

Za předpokladu tříhladinového modelu fluorescenčního procesu uvedeného výše
lze při určování koncentrace vyjít z~rychlostní rovnice popisující
koncentraci horního stavu $\ndens_3$:
\begin{equation}
	\label{eq:lifth-rate-simple}
	\dv{\ndens_3}{\tim}
	= \specoverlap \frac{\einsteinb13 \ity}{\lightspeed} \ndens_1
	- \specoverlap \frac{\einsteinb31 \ity}{\lightspeed} \ndens_3
	- ( \einsteina32 + \quenching_{32} ) \ndens_3,
\end{equation}
kde $\ndens_1$ je koncentrace základního stavu,
$\einsteina{i}{j}$ je Einsteinův koeficient spontánní emise,
$\einsteinb{i}{j}$ jsou Einsteinovy koeficienty absorpce,
$\quenching_{ij}$ je koeficient zohledňující zhášení,
$\ity$ je intenzita ozáření laserem
a~$\lightspeed$ je rychlost světla.
Kromě toho vystupuje ve vztahu také takzvaný
\emph{spektrální překryv} $\specoverlap$,
definovaný pomocí profilu absorpční čáry $a$
a~spektrálního složení laserového záření $l$ normovaných na jedničku:
\begin{equation}
	\label{eq:lifth-specoverlap}
	\specoverlap = \int l(\freq)\,a(\freq) \dd{\freq}.
\end{equation}

Obvykle se mezi základní a~horní hladinou nacházejí další stavy,
které je potřeba do úvahy zahrnout.
Vztah~\eqref{eq:lifth-rate-simple} potom nabývá podoby:
\begin{equation}
	\label{eq:lifth-rate}
	\dv{\ndens_3}{\tim}
	= \specoverlap \frac{\einsteinb13 \ity}{\lightspeed} \ndens_1
	- \specoverlap \frac{\einsteinb31 \ity}{\lightspeed} \ndens_3
	- \left( \sum_i \einsteina3i + \sum_i \quenching_{3i} \right) \ndens_3,
\end{equation}
kde sumy probíhají přes všechny mezilehlé stavy $i$.
Výraz v~závorce udává rychlost úbytku horního stavu všemi samovolnými
procesy (tedy těmi, které nejsou způsobeny laserovým zářením).
Převrácená hodnota této veličiny se nazývá \emph{doba života} $\lifetime$:
\begin{equation}
	\label{eq:lifth-lifetime-def}
	\lifetime = \frac{1}{\sum_i \einsteina3i + \sum_i \quenching_{3i}}.
\end{equation}

Za předpokladu, že se po odeznění laseru vrátí soustava do původního stavu,
musí být celková změna koncentrace $\ndens_3$ rovna nule.
Integrací vzta\-hu~\eqref{eq:lifth-rate} podle času lze tedy získat rovnici:
\begin{equation}
	\label{eq:lifth-rate-int}
	0 = \frac\specoverlap\lightspeed
	\int_0^\infty (\einsteinb13 \ndens_1 - \einsteinb31 \ndens_3) \ity\,\dd\tim
	- \frac1\lifetime \int_0^\infty \ndens_3 \dd{\tim}.
\end{equation}
Celkové množství fluorescenčního záření lze vyjádřit jako:
\begin{equation}
	\label{eq:lifth-liftotal-def}
	\liftotal = \einsteina32 \int_0^\infty \ndens_3 \dd{t}.
\end{equation}
Po dosazení tohoto výrazu do rovnice \eqref{eq:lifth-rate-int}
lze celkové vyzářené množství vyjádřit jako:
\begin{equation}
	\label{eq:lifth-liftotal-general}
	\liftotal = \einsteina32 \lifetime \frac\specoverlap\lightspeed
	\int_0^\infty
	(\einsteinb13 \ndens_1 - \einsteinb31 \ndens_3)
	\ity\,\dd\tim.
\end{equation}
Součin $\einsteina32 \lifetime$ se nazývá kvantový výtěžek a~popisuje,
jaký podíl přechodů se podílí na fluorescenčním záření.

Za předpokladu, že množství stimulované emise do základního stavu
je výrazně menší než absorpce laserového záření
a~že celkové množství popsaných jevů je dostatečně malé, aby mohla být
koncentrace základního stavu $\ndens_1 = \ndens$ považována za konstantní,
lze výraz \eqref{eq:lifth-liftotal-general} zjednodušit následovně:
\begin{equation}
	\label{eq:lifth-liftotal-linear}
	\liftotal \approx \einsteina32 \lifetime
	\frac{\specoverlap\einsteinb13}{\lightspeed}
	\ndens
	\int_0^\infty \ity\,\dd\tim
	\qquad
	\text{kde} \int_0^\infty \ity\,\dd\tim \propto \enlaser,
\end{equation}
kde je možno využít přímé úměry mezi integrálem intenzity
a~energií laserového pulzu $\enlaser$.

Intenzita signálu snímaná v~tomto lineárním režimu je potom
integrálem fluorescenčního záření přes celý objem $\vol$:
\begin{equation}
	\label{eq:lifth-lifsignal}
	\lifsignal = \einsteina32\,\lifetime
	\frac{\specoverlap \einsteinb13}{\lightspeed}
	\ndens \enlaser \lifeff
	\iiint_\vol \sensabs \frac{\solidangle}{4\pi} \beamprofile \dd{\vol},
\end{equation}
kde bylo zavedeno několik parametrů popisujících uspořádání:
$\cameraangle$ je prostorový úhel fluorescenčního záření dopadajícího
na snímač
a~$\beamprofile$ je prostorový profil intenzity laseru normovaný na jedničku.

Citlivost snímače vůči fluorescenčnímu záření ($\lifsens$) obvykle není přímo
známa,
a~byla proto rozepsána jako součin dvou veličin $\lifeff$ a~$\sensabs$.
Relativní citlivost $\lifeff$ je funkcí vlnové délky a~běžně bývá dodávána
výrobcem; mnohdy představuje kvantovou účinnost detektoru.
Absolutní citlivost $\sensabs$ nezávisí na vlnové délce,
ale její hodnota je zpravidla neznámá nebo jen obtížně stanovitelná.

Z~výše uvedeného plyne nutnost při měření zaznamenat nejen samotnou
intenzitu fluorescence $\lifsignal$,
ale také dobu života zářivého stavu $\lifetime$.
Tu je možno změřit z~časového vývoje fluorescenčního záření po odeznění
laserového pulzu,
neboť doba života definovaná vztahem~\eqref{eq:lifth-lifetime-def} vystupuje
jako parametr exponenciálního poklesu intenzity fluorescence v~čase:
\begin{equation}
	\lifsignal \propto \eu^{-\frac{\tim}{\lifetime}}.
\end{equation}
Dobu života lze tedy získat proložením časového vývoje fluorescenčního
signálu exponenciální fukncí.

Větší potíže působí objemový integrál na konci, jehož hodnota není snadno
vyčíslitelná.
Obvykle se proto přistupuje ke kalibraci pomocí Ray\-leigh\-ova rozptylu
ve známém médiu (například vzduchu),
která umožní tento integrál eliminovat.
Měřenou intenzitu Rayleighova rozptylu $\rayleighsignal$ lze popsat vztahem:%
\autocite{lif-oh}
\begin{equation}
	\label{eq:lifth-rayleighsignal}
	\rayleighsignal = \rayleighdxsect \, \rayleighndens
	\frac{\enlaserrayleigh}{\planck\freq} \, \rayleigheff
	\iiint_\vol \sensabs \solidangle \beamprofile \dd{\vol},
\end{equation}
kde $\rayleighndens$ je koncentrace rozptylujících částic,
$\enlaserrayleigh$ je energie laserového pulzu,
$\freq$ je frekvence laserového záření,
$\rayleigheff$ je kvantová účinnost snímače pro použitou vlnovou délku
a~$\rayleighdxsect$ je diferenciální účinný průřez Rayleighova rozptylu.

V~případě vzduchu lze koncentraci částic přibližně vyjádřit pomocí
stavové rovnice ideálního plynu:
\begin{equation}
	\rayleighndens = \frac{\pres}{\boltzmann\temp}.
\end{equation}
Stanovení účinného průřezu $\rayleighdxsect$ je obtížnější,
ale pro běžná média je dobře popsáno v~literatuře.
Rayleighově rozptylu ve vzduchu se věnuje například Miles v~\cite{rayleigh},
kde udává hodnoty indexu lomu a~diferenciálního účinného průřezu vzduchu
pro vlnové délky v~rozsahu \SIrange{200}{1000}{\nano\metre}.

Kombinace vztahů \eqref{eq:lifth-lifsignal} a~\eqref{eq:lifth-rayleighsignal}
vede (v~lineárním režimu) na následující vyjádření koncentrace částic,
kde již objemový integrál nevystupuje:
\begin{equation}
	\label{eq:lifth-ndens}
	\ndens = \frac{\lifsignal}{\rayleighsignal}
	\frac{\enlaserrayleigh}{\enlaser}\frac{\rayleigheff}{\lifeff}
	\frac{1}{A_{32}\,\tau}
	\frac{\lightspeed}{\planck\freq\specoverlap B_{13}}
	\, 4\pi \rayleighdxsect \frac{p}{kT}.
\end{equation}

\section{Saturace}
\label{sec:lifth-saturation}
Podmínky pro dosažení lineárního režimu však nemusejí být vždy splněny.
Při zvýšení intenzity laseru nad určitou hodnotu přestanou platit oba
výše uvedené předpoklady:
Začne se projevovat vyčerpání základní hladiny $\ndens_1$
a~laserem stimulovaná emise zpět do základního stavu nabude nezanedbatelné
hodnoty.
Dostatečná energie laseru navíc může vést k~fotoionizaci excitovaného stavu.
Konečným důsledkem je, že intenzita fluorescenčního záření je nižší,
než předpovídá lineární model.
Tomuto jevu se říká saturace.

Saturace ztěžuje vyhodnocení experimentálních dat, neboť pro stanovení
koncentrace již není možno použít výše popsaný lineární model
a~odvozený vztah~\eqref{eq:lifth-ndens}.
Existují však možnosti, jak jej pro saturaci korigovat.

\Citeauthor{lif-saturation} odvodili v~roce \citeyear{lif-saturation}
následující vztah pro přibližný popis částečně saturovaného
fluorescenčního jevu\autocite{lif-saturation}:
\begin{equation}
	\liftotal(\enlaser) = \frac{\lifslope\enlaser}{1 + \lifsat\enlaser}.
\end{equation}
Zde $\lifslope\enlaser$ je hypotetická fluorescence bez účinků saturace
a~$\lifsat$ je takzvaný saturační parametr.
V~navazující studii z~roku \citeyear{lif-pb} uvádí \citeauthor{lif-pb},
že saturaci lze ve vztahu \eqref{eq:lifth-ndens} zohlednit,
pokud se podíl $\lifsignal/\enlaser$ nahradí vhodným výrazem
závisejícím na konkrétním provedení experimentu.
Zároveň doplňuje příklady pro dvě nejběžnější uspořádání.%
\autocite{lif-pb}

Pro úzký rovinný svazek pozorovaný z~kolmého směru má výraz tvar:
\begin{equation}
	\label{eq:lifth-sheet}
	\liftotal = \frac{2\lifslope}{\lifsat}
	\left( 1 - \frac{\ln(1 + \lifsat\enlasery)}{\lifsat\enlasery} \right),
\end{equation}
kde $\enlasery$ je lineární hustota laserové energie podél osy $y$.
Pro osově souměrný gaussovský svazek nabývá výraz podoby:
\begin{equation}
	\label{eq:lifth-cylindric}
	\liftotal = \frac{\lifslope}{\lifsat}
	\ln(1 + \lifsat\enlaser),
\end{equation}
kde $\enlaser$ je celková energie laserového pulzu.
