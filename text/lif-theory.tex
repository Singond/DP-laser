\chapter[LIF]{Laserem indukovaná fluorescence}
\label{sec:lifth}
Počátky laserem indukované fluorescence (LIF) sahají do roku 1968,
kdy Richard Zare pomocí \num{632.8}\si{\nano\metre} čáry helium-neonového
laseru analyzoval částice v~draslíkových výparech.
\autocite{lif-original}
Od té doby se s~úspěchem používá ke studiu průhledných médií,
jako jsou plameny nebo plazma.
Ve fyzice plazmatu hraje klíčovou roli při detekci reaktivních částic.
% TODO: \autocite{dvorak1}

Metoda v~sobě spojuje prvky absorpční a~emisní spektroskopie.
Je sice obecně složitejší a~vyžaduje rozsáhlejší a~dražší přístrojové vybavení
než obě tyto metody, ale na oplátku poskytuje několik zásadních výhod.
Mezi nejdůležitější patří nízký detekční limit
a~schopnost detekce nezářivých částic nebo částic s~krátkou dobou života.
Velmi užitečné je také vysoké prostorové rozlišení metody:
Pokud je laserový svazek zaostřen do úzkého profilu a~pozorován z~boku,
je možné získat signál rozlišený ve všech třech prostorových rozměrech.
Při použití krátkopulzního laseru navíc umožňuje výborné časové rozlišení
v~řádu nanosekund až femtosekund.
\autocite{lif-pb}

\section{Princip}
Základní myšlenka metody je pozorování fluorescenčního záření vznikajícího
při deexcitaci zkoumaných částic ze stavu vybuzeného absorpcí laserového
světla.
Obecné schéma nejjednodušší varianty je na obrázku \ref{fig:lifth-levels}.
Atomy či jiné částice jsou dopadem laseru excitovány ze~základní hladiny~1
do excitovaného stavu~3.
Tento stav je depopulován jednak samovolnou emisí zpět do základního stavu,
jednak vynucenou emisí do základního stavu
a~samovolnou emisí do stavu~2.
Záření přechodu z~hladiny~3 na hladinu~2 je snímaný fluorescenční signál.

\begin{figure}
	\centering
	\begin{tikzpicture}[scale=0.5]
		\small
		\lifgrotrian
	\end{tikzpicture}
	\caption{Obecné excitační schéma jednofotonové LIF.
		Parametry $\einsteina{i}{j}$ a~$\einsteinb{i}{j}$ jsou Einsteinovy
		koeficienty a~$I$ je intenzita záření.
		Kromě samovolné emise jsou vyšší stavy depopulovány zhášením
		(koeficienty $Q_{ij}$).
		Podle \cite{lif-pb}.}
	\label{fig:lifth-levels}
\end{figure}

\begin{figure}[htb]
	\centering
	\includegraphics[width=\textwidth]{lif-setup-general}
	\caption{Příklad uspořádání experimentu s~LIF.
		Laserový svazek je v~tomto případě pomocí válcových rozptylek
		rozšířen do rovinného tvaru, který je zboku snímán kamerou.
		Podle \cite{lif-oh}.}
	\label{fig:lifth-setup}
\end{figure}

\section{Určení koncentrace částic}
\label{sec:lifth-concentration}
\providecommand\vol{V}
\providecommand\solidangle{\Omega}
\providecommand\sensabs{D_\text{a}}
\providecommand\lifsens{D_\text{F}}
\providecommand\rayleighsens{D_\text{R}}
\providecommand\lifsignal{M_\text{F}}
\providecommand\rayleighsignal{M_\text{R}}
\providecommand\lifeff{\qeff_\text{F}}
\providecommand\rayleigheff{\qeff_\text{R}}
\providecommand\rayleighdxsect{\dv{\sigma_\text{R}}{\solidangle}}
\providecommand\rayleighndens{\ndens_\text{R}}
\providecommand\enlaserrayleigh{L_\text{R}}
\providecommand\beamprofile{s}
\providecommand\cameraangle{\solidangle}
Velkou výhodou LIF je možnost určit absolutní koncentraci detekovaných částic,
neboť intenzita fluorescence je funkcí této koncentrace.
Závislost je v~limitě nízkých energií laseru lineární,
s~rostoucí energií se pak objevují účinky saturace,
které fluorescenci oslabují (viz dále v~oddíle \ref{sec:lifth-saturation}).

Za předpokladu termodynamické rovnováhy rotačních stavů uvnitř každé
vibrační hladiny lze intenzitu signálu $\lifsignal$ (pro jeden pozorovaný
přechod) vyjádřit pomocí hustoty částic $\ndens$ v~základním stavu:
\begin{equation}
	\label{eq:lifth-lifsignal}
	\lifsignal = \einsteina32\,\lifetime
	\frac{\specoverlap \einsteinb13}{\lightspeed}
	\ndens \enlaser \lifeff
	\iiint_\vol \sensabs \frac{\solidangle}{4\pi} \beamprofile \dd{\vol}.
\end{equation}
Zde $\enlaser$ je energie laserového pulzu,
$\lifetime$ je doba života zářivého stavu,
$\cameraangle$ je prostorový úhel fluorescenčního záření dopadajícího
na snímač,
$\beamprofile$ je prostorový profil intenzity laseru normovaný na jedničku,
$\einsteina{i}{j}$ a~$\einsteinb{i}{j}$ jsou Einsteinovy koeficienty,
a~$\lightspeed$ je rychlost světla.
Kromě toho vystupuje ve vztahu také takzvaný
\emph{spektrální překryv} $\specoverlap$,
definovaný pomocí profilu absorpční čáry $a$
a~spektrálního složení laserového záření $l$ normovaných na jedničku:
\begin{equation}
	\specoverlap = \int l(\freq)\,a(\freq) \dd{\freq}.
\end{equation}
Citlivost snímače na fluorescenční záření ($\lifsens$) byla rozepsána
jako součin kvantové účinnosti snímače $\lifeff$ a~neznámé
(nebo jen obtížně stanovitelné) konstanty $\sensabs$,
protože relativní kvantová účinnost $\qeff$ pro různé vlnové délky
je obvykle známa.

Z~výše uvedeného plyne nutnost při měření zaznamenat nejen samotnou
intenzitu fluorescence $\lifsignal$,
ale také dobu života zářivého stavu $\lifetime$.
Tu je možno změřit z~časového vývoje fluorescenčního záření,
neboť doba života je definována jako parametr exponenciálního poklesu
intenzity fluorescence:
\begin{equation}
	\lifsignal \propto \eu^{-\frac{\tim}{\lifetime}}.
\end{equation}

Větší potíže působí objemový integrál na konci, jehož hodnota není snadno
vyčíslitelná.
Obvykle se proto přistupuje ke kalibraci pomocí Ray\-leigh\-ova rozptylu
ve známém médiu (například vzduchu),
která umožní tento integrál eliminovat.
Měřenou intenzitu Rayleighova rozptylu $\rayleighsignal$ lze popsat vztahem:
\begin{equation}
	\label{eq:lifth-rayleighsignal}
	\rayleighsignal = \rayleighdxsect \rayleighndens
	\frac{\enlaserrayleigh}{\planck\freq} \rayleigheff
	\iiint_\vol \sensabs \solidangle \beamprofile \dd{\vol},
\end{equation}
kde $\rayleighndens$ je koncentrace rozptylujících částic,
$\enlaserrayleigh$ je energie laserového pulzu,
$\freq$ je frekvence laserového záření,
$\rayleigheff$ je kvantová účinnost snímače pro použitou vlnovou délku
a~$\rayleighdxsect$ je diferenciální účinný průřez Rayleighova rozptylu.

V~případě vzduchu lze koncentraci částic přibližně vyjádřit pomocí
stavové rovnice ideálního plynu:
\begin{equation}
	\frac{\pres}{\boltzmann\temp}.
\end{equation}

% TODO: účinný průřez

Kombinace vztahů \eqref{eq:lifth-lifsignal} a~\eqref{eq:lifth-rayleighsignal}
vede na následující vyjádření koncentrace částic,
kde již objemový integrál nevystupuje:
\begin{equation}
	\label{eq:label}
	\ndens = \frac{\lifsignal}{\rayleighsignal}
	\frac{\enlaserrayleigh}{\enlaser}\frac{\rayleigheff}{\lifeff}
	\frac{1}{A_{32}\,\tau}
	\frac{\lightspeed}{\planck\freq\specoverlap B_{13}}
	\, 4\pi \rayleighdxsect \frac{p}{kT}.
\end{equation}

\section{Saturace}
\label{sec:lifth-saturation}
